\documentclass[12pt, oneside, a4paper]{article}
\usepackage{ifpdf}
\usepackage[colorlinks,bookmarksopen]{hyperref}
\usepackage[utf8]{inputenc}
\usepackage[english,russian]{babel}
\usepackage{amsmath}
\usepackage{booktabs}
\usepackage{array}
\everymath{\displaystyle}
\begin{document}
\section*{Введение}2
В 1864 г. Максвелл, на основе его известных уравнений, предположил, что свет представляет собой поперечные электромагнитные волны. Несмотря на это, сам Максвелл не рассматривал возможность получить свет электромагнитными методами. Впрочем, ничего конкретного, о самих электромагнитных волнах, их генерации и детектировании, в его работах нет. Только спустя почти четверть века с момента опубликования электромагнитной теории света Максвелла, Генрих Герц (Heinrich Hertz) в своих выдающихся  экспериментах зарегестрировал и детально изучил основные свойтсва электромагнитных волн, подтвердив, тем самым, теорию Максвелла. Идеи и уравнений  Максвелла были детально разработаны, модифицированы и облачены в удобную для понимания их физической сути форму усилиями Фитцжеральда (G.F.~FitzGerald), Оливера Лоджа (Oliver Lodge), Оливера Хевисайда (Oliver Heaviside) и Генриха Герца. Первые три ученых образовывали группу, неформально названную Максвеллианцами. 

Данная работа посвящена описанию становления электромагнитной теории света Максвелла как одной из самых ярких обобщающих физических теорий.
\section*{Ранние теории и гипотезы}
Задолго до Максвелла, многие ученые и натурфилософы пытались объяснить, каким образом действие электрическии и магнитных  сил распространяется в пространстве. Карл Фрейдрих Гаусс, ``принц математики'', в 1855 выдвинул идею об ограниченной скорости распространения электрического взаимодействия между зарядами, но решил не публиковать свою работу, поскольку не смог построить модель в среды, в котором такое распространение было бы возможным. Неоднократные попытки развить идеи Гаусса были предприняты его учеником, Риманом, который в 1853 г., основываясь на уравнении Пуассона для электростатического потенциала, получил волновое уравнение, согласно которому, изменение заряда влечет за собой возмуцение потенциала, которе распространяется в пространстве со скоростью света. Несмотря на то, что это предположение согласуется с современной теорией электромагнетизма, гипотеза Римана была тривиальной, чтобы рассматриваться как основа полноценной теории. В 1867 г. в \emph{Анналах} Поггендорфа появились две работы. Одна из этих работ, так и не опубликованная, принадлежала Бернхарду Риману, который в 1858 г. показал, что из обобщенного уравнения Лапласа возможно получить волновое решение. Во второй работе, Лоренц показал, что в теории Вебера, периодические возмущения электрического заряда способны распространяться со скоротью света.

В депозите Королевского общества (The Royal Society) находится работа, озаглавленная ``The Original Views'', в которой Майкл Фарадей выступает с идеей, что эффекты, связанные с электричеством и магнетизмом ``не проявляются мгновенно и распространяются с конечной скоростью''. Фарадей не нашел времени, чтобы провести экспериментальную проверку своей гипотезы, и поэтому пожелал сохранить свою работу 1832 года в депозите, с завещанеим не публиковать ее минимум 100 лет.
\section*{Максвелл}
Джеймс Клерк Максвелл (James Clerk Maxwell, 1831-1879 гг.) был одним из выдающихся физиков девятнадцатого столетия. Спектр проблем, охватываемый его работами, простирается от кинетической теории газов до теории цветового восприятия, но наибольшую известность принесла Максвеллу его электромагнитная теория света. Несмотря на то, что собрания сочинений Максвелла и исследования его работ несколько раз переиздавались, большинство бумаг, относящиеся непосредственно к личной жизни Максвелла, были утеряны после его смерти. Наиболее полная, на данный момент, биография Максвелла быда издана в 1882 году.

Максвелл родился и вырос в Эдинбурге, в семье, происходившей из юго-западной Шотландии. После окончания Эдинбургского университета, Максвелл поступает в Кембридж, который закончивает в январе 1854 г. с отличием в математике. В поисках темы для своей исследовательской работы, в следующем месяце он пишет письмо Томсону, в котором рассказывате о своем интересе к теории электричества и просит рекомендаций относительно работ Фарадея. Неизвестно, почему Максвелл выбрал электричестсво, тему, которая была исключена из учебного курса Кемриджа, и почему он примкнул к неортодоксальному подходу Фарадея к этой задаче, но стоит заметить, что  его письмо Томсону появляется на волне интереса, поднятого лекцией Фарадея о затухании электромагнитного сигнала в кабеле, данной в Королевском Институте (Royal Institution). 

Первым результатом исследований Максвелла была работа ``О фарадевских силовых линиях'', законченная в начале 1856. Проводя аналогию между силовыми линиями электрического и магнитного полей и потоками в жидкость, Максвелл облек расплывчатые формулировки Фарадея в строгую математическую форму. В 1861 он выступил с амбициозной работой ``О физических силовых линиях'', основанной на разработанной им механической модели эфира, состоящего из крошечных вихрей и шестеренок. Эта модель привела Максвелла не только к уравнению, связывающему основные электромагнитные величины и пониманию, что переменные электрические силы производят ток смещения, но также к удивительному выводу, что \emph{свет состоит в поперечных колебаниях той же среды, которая является переносчиком электрических и магнитных явлений}. 
Электромагнитная теория света по праву считается одной из величайших обобщающих теорий во всей физике, поэтому путь, которым Максвелл пришел к ней, детально изучен. Единственный вопрос, до сих пор вызывавший споры среди историков и философов науки --- это вопрос о том, насколько реальной Максвелл считал свою модель, изложенную в ``Физических линиях''. Общим мнением считается, что Максвелл действительно рассматривал вихри как реально существующие в магнитном поле, однако, промежуточные шестерни были введены им не более чем для удобства и наглядности теоритеческого рассмотрения. 

Лучшим доказательством электромагнитной теории Максвелла было совпадение между скоростью света, измеренным Физо (Hippolyte Fizeau) и другими, и отношением между элеткрическими и магнитными системами единиц, измеренным Вебером (Weber) и Рудольфом Колраушем (Rudolph Kohlraush). Более точное измерение отношения единиц должно было стать серьезным испытанием для теории Максвелла, и, надежде на это, он вступает в Британский Комитет по Электрическом Стандартам (British Committee on Electrical Standarts) в 1862 г. Затем, будучи уже профессором в Королевском Лондоском Колледже (Kings College London), Максвелл в течении последующих двух лет тесно работает с инженером-кабельщиком Флемингом Дженклином (Fleeming Jenklin) над вычислением искомого соотношения, используя методику, разработанную Томсоном. Максвелл вычисляет соотношение, и, несмотря на некоторые несостыковки, заключает, что оно находится соответсвии со скоростью света, достаточным для того, чтобы убедиться в правоте собственной теории, несмотря на то, что Томсон не принял это доказательство как решающее. 

В декабре 1864 г. Максвелл представляет Королевскому Обществу (Royal Society) свою ``Динамическую теорию электромагнитного поля'', в которой он выводит электромагнитные уравнения не используя какую-либо механическую модель, как в ``Физических линиях'', но из общей динамики согласованной системы. Основаясь на фарадеевских представлениях о зарядах и токах как о вторичных свидетельствах состояния окружающих систему полей, Максвелл выражает это состояние через изменение того, что он называет ``электромагнитным моментом'' (позже переименнованым в вектор потенциал), и описывает, как распределяется энергия в электромагнитном поле. Максвелл по-прежнему верит в существование эфира, но, до тех пор, пока до конца не выяснены детали его внутреннего устройства, Максвелл считает, что лучше формулировать теорию, имаользуя минимимум предположений. 

Максвелл покинул Королевский Колледж в 1856 г. и провел следующие несколько лет в своем поместье в Шотландии за написанием  \emph{Трактата об электричестве и магнетизме} (1873 г.). Несмотря на что, оно было полно ценными идеями, труд, в целом, получился бессвяным и сложным для понимания. К тому времени, как \emph{Трактат} был опубликован, Максвелл вступает в основанное в 1871 г. кавендишское профессорство в области экспериментальной физики, в Кембридже, и занимается организацией новой лаборатории, основной деятельностью которой он рассматривает проведение прецизионных электрических измерений. Максвелл умер от рака 5 ноября 1879, в возрасте 48 лет, во время работы над переизданием \emph{Трактата}. К сожалению, к этому времени, его электромагнитная теория света не была ни полностью понятой, ни широко распространненой, более того, в некотором смысле, она не была законченной.
\section*{Теория Максвелла}
Впервые, уравнения Максвелла, связывающие воедино, в математической форме, всё известное на том момент об электромагнетизме, появляются в работе 1864 г. ``Динамическая теория электромагнитного поля''. В этой работе Максвелл представляет 20 уравнений, содержащих 20 неизвестных.  В своих уравнениях Максвелл суммирует труда Эрстеда (1777-1851), Карла Гаусса (1777-1855), Ампера (1775-1836), Фарадея (1791-1867) и других, а так же добавляет фундаментальное понятие ``тока смещения''.

Для дальнейшего изложения полезно будет рассмотреть  уравнения Максвелла, в том виде, в котором он изложил их в своей работе. В таблице [\ref{tab:maxwell_vars}] приведены обозначения физических велични, использованных Максвеллом и их современная нотация. 

Наборы из трех величин, представленные в среднем чтолбце первых шести строчек таблицы [\ref{tab:maxwell_vars}], представляют собой декартовы компоненты ${x,\,y,\,z}$ соответствующих векторов в левом крайнем столбце. Помимо переменных, указанных в таблице, Максвелл неявно пользуется ``магнитной индукцией'', трик компонеты вектора которой, в изотропной среде, он обозначает как $\mu_\alpha,\,\mu_\beta,\,\mu_\gamma$. Современное нзвание для этой величны --- вектор напряженности магнитного поля, $\mathbf{B}=\mu\mathbf{H}$, где $\mu$ определяется средой.
\begin{table}[p]
\begin{tabular}{>{\raggedright}m{4cm}>{\centering}m{3.5cm}>{\centering}m{3.5cm}}
\toprule
Название физической величины & Обозначения Максвелла & Современная нотация\tabularnewline
\midrule
Вектор потенциал & $F,\,G,\,H$ & $\mathbf{A}$\tabularnewline\tabularnewline
Индукция магнитного поля & $\alpha{},\,\beta{},\,\gamma{}$ & $\mathbf{H}$\tabularnewline 
\tabularnewline
Напряженность электрического поля & $P,\,Q,\,R$ & $\mathbf{E}$\tabularnewline
\tabularnewline
Ток проводимости & $p,\,q,\,r$ & $\mathbf{J}$\tabularnewline
\tabularnewline
Ток смещения & $f,\,g,\,h$ & $\mathbf{D}$\tabularnewline
\tabularnewline
Полный ток & 
\[
\begin{Bmatrix} 
p^\mathrm{l}=p+\frac{\mathrm{d}f}{\mathrm{d}t}\\
\\
q^\mathrm{l}=q+\frac{\mathrm{d}g}{\mathrm{d}t}\\
\\
r^\mathrm{l}=r+\frac{\mathrm{d}h}{\mathrm{d}t}
\end{Bmatrix}
\]
& $\mathbf{J_T}$\tabularnewline
\tabularnewline
Плотность свободного заряда & $e$ & $\rho$\tabularnewline
\tabularnewline
Электрический потенциал & $\psi$ & $\psi$\tabularnewline
\bottomrule
\end{tabular}
\caption{Оригинальные обозначения, использованные Максвеллом в своих уравнениях}
\label{tab:maxwell_vars}
\end{table}

Используя указанные 20 переменных, приведенных в таблице [\ref{tab:maxwell_vars}], Максвелл, в приближении изотропной среды, выписывает уравнения, формирующие базис ``Динамической теории электромагнитного поля'', записанные здесь, для удобства, в векторной форме с использование современных обозначений.
\begin{align}
&\mathbf{J_T}=\mathbf{J}+\frac{\partial{\mathbf{D}}}{\partial{t}}\label{eq:A}\\
&\mu\mathbf{H} = \mathbf{B} = \nabla\times\mathbf{A}\label{eq:B}\\
&\nabla\times\mathbf{H}=4\pi\mathbf{J_T}=4\pi\left[\mathbf{J}+\frac{\partial{\mathbf{D}}}{\partial{t}}\right]\label{eq:C}\\
&\mathbf{E}=\boldsymbol\nu\times\mathbf{B} - \frac{\partial{\mathbf{D}}}{\partial{t}} - \nabla\psi\label{eq:D}
\end{align}
Уравнение~\eqref{eq:D} названо Макселлом уравнением для электродвижущей силы в проводнике, движущимся со скоростью $\boldsymbol\nu$ в изотропной среде.
\begin{align}
\mathbf{E}=\mathrm{k}\mathbf{D},\label{eq:E}
\end{align}
где $\mathrm{k}$ назван Максвеллов ``коэффициентом электрической эластичности''. Для сравнения, современная запись уравнения~\eqref{eq:E} выглядит как $\mathbf{D}=\varepsilon\mathbf{E}$, $\varepsilon$---проницаемость среды.
\begin{align}
\mathbf{E}=\rho^{'}\mathbf{J},\label{eq:F}
\end{align}
где $\rho^{'}$ характризует ``сопротивляемость'' или, иначе, сопротивление среды. Максвелл использовал символ $\rho$ вместо $\rho^{'}$, однако, чтобы не вступать в конфликт с обозначением для объемной плотности заряда, здесь использутеся $\rho^{'}$. В современной традиции, уравнение~\eqref{eq:F} записывается как $\mathbf{J}=\sigma\mathbf{E}$, где $\sigma=1/\rho^{'}$.
\begin{align}
&\nabla\cdot\mathbf{D}=\rho\label{eq:G}\\
&\nabla\cdot\mathbf{J}+\frac{\partial{\rho}}{\partial{t}}=0\label{eq:H}
\end{align}
\section*{Технический прогресс и теория Максвелла}
В вводном слове к своему \emph{Трактату об электричестве и магнетизме}, Максвелл отмечает, что 
\begin{quote}
\small
многочисленные приложения, которые нашла теория электромагнетизма в телеграфии, в конечном итоге, повлияли на развитие чистой науки в том смысле, что составили коммерческую ценность точным электрическим измерениям и предоставили заинтересованным ученым техническое оснащение в масштабах и количестве, недостижимом для рядовой лаборатории. Подобная заинтересованность в знаниях об электричестве и, одновременно, экспериментальная возможность получить их, уже дали свои первые плоды, как в стимулировании усилий людей науки, так и повсеместном распространении точного знания среди практиков, что, в целом, способствовало  всеобщему научному прогрессу во всей инженерной профессии.
\end{quote}
Максвелл, очевидно, подразумевает здесь свою работу в Комитете по Стандартам и, конечно же, участие Томсона в работе над проектом прокладки второго Атлантического кабеля, законченную в 1866 г.

Спустя несколько лет после провала в 1858 г., Филд и партнеры предпринимают вторую попытку проложить кабель по дну Атлантического океана. Следуя рекомендациям, данным Томсоном, уже более последовательно в этот раз, партнеры заказывают более тонкий кабель и тщательно его проверяют. В июле 1865 корабль \emph{Great Eastern} --- единственный на тот момент во всем флоте, способный нести на борту такое количество кабеля (4260 км), начинает прокладку от Ирландии, но, к сожалению, кабель обрывается, когда было проложено около 1968 км, а его конец теряется. Проект кажется обреченным на провал, но, при поддержке Джона Пендера, богатого торговца хлопком из Манчестера, Филд и партнеры заказывают еще один кабель и следующим летом предпринимают уже третью попытку. В этот раз все проходит гладко и 27 июля 1866 г. партнеры закрепляют кабель вблизи Ньюфаундленда. Вскоре после этого, \emph{Great Eastern} находит ранее утеряный конец кабеля, оставшийся от неудачной попытки 1865 г., сращивает его и заканчивает прерванные работы по прокладке, и, таким образом, с сентября 1866 г. Атлантика стянута сразу двумя рабочими кабелями.

Успех 1866 г. вызвал всеобщий бум в прокладке кабеля. К 1875 г., британские компании (владельцем большинства из них был тот самый Пендер), проложили кабели к Индии, Австралии, Гонг-Конгу; к 1890 кабели опоясывают побережья Южной Америки и Африки, и обеспечивают связью уже весь мир. Телеграфия становится огромной и прибыльной индустрией, сильно заинтересованной в квалифицированных электриках в период 1850-1880 гг. Британские фирмы владели на тот момент примерно двумя третями от общей длины проложенного кабеля, что способствовало повышенной заинтересованности британских ученых в исследовании вопросов распространения электромагнитных сигналов в линиях.

В конце 1870-х, стали появляться первые электрические станции, ознаменовав, тем самым, рождение электротехнической промышленности, и вызвав новую волну спроса на исследования в области эдектричества. Электроэнергия, получаемая от батерей Вольта, или от генераторов на постоянных магнитах, была слишком слабой и дорогостоющей для широкого использования. В 1866-1867 гг. несколько изобретателей предлогают использовать в генераторах электромагниты, что существенно увеличивает их мощность и делает электроэнергию, на первое время, относительно дешевой и доступной. Первым основным применением электроэнергии, получаемой от динамо-машин, были осветительные дуги, но скоро последние стали слишком яркими для использования их в помещениях. Томас Эдисон (Thomas Edison, 1816-1890), изобретает лампу накаливания в 1879 и приступает к разработке системы производства и распределения электрической мощности; в 1882 г. он устонавливает первые электрические станции на Pearl Street в Нью-Йорке.

Эдисона разработал систему распределения постоянного тока, которая подходила для передачи электроэнергии только на короткие расстояния вследствие сильных потерь в линии. Системы распределения переменного тока были лишены этого недостатка, используя повышащие трансформаторы перед передачей электроэнергии на дальние расстояния, и понижающие --- для окончательного распределения мощности между местными потребителями. В нашумевшей ``битве систем'', длившейся вплоть до 1890 г., победителями вышли системы переменного тока.

Ранние системы распределения постоянного тока были примитивными, электрический ток в их проводах мог, в большинстве случаев, рассматриваться как поток воды в трубопроводе. Системы распределения переменного тока были уже тогда более сложными (в частности, после изобретения метода полифазной передачи), в которых наблюдались многие эффекты, связанные с переменными электромагнитными полями. Стремительный рост электротехнической промышленности в 1880-х и 1890-х годах способствовал безпрецендентному росту потребности в квалифицированных инженерах-электриках, компетентых для работы с переменным током. Этот спрос нашел отклик в физических факультетах университетов по всему миру, которые стали открывать расширять свои кафедры и лаборатории, чтобы справиться с нахлынувшим потоком студентов. Можно с уверенностью сказать, что развитие физики как дисциплины во второй половине девятнадцатого столетия во многом связан со спросом и возможностями, предоставляемыми электротехнической промышленностью. 
\section*{Максвеллианцы}
Теория Максвелла, в том виде, в которм он оставил ее после своей смерти в 1879 г., рассматривалась научным сообществом того времени как одна из многих попыток объяснить распространение света в пространстве и, тем более, никоим образом не претендовала на лидерство в своей области. Но уже к 1890 г. она смела своих конкурентов и, по-праву, заняла место одной из самых удачных и фундаментальных теорий во всей физике. Это стало возможным, во многом, благодаря работам Генриха Герца и группы ``Максвеллианцев'', среди которых особо выделялись Фитцжеральд, Лодж и Хевисайд. Эти ученые, на протяжении 1880-х гг., переформулировали теорию в более четком и компактном виде, подвергли ее экспериментальной проверке  и обобщили на случаи, которые сам Максвелл не смог предвидеть. В ходе этой работы, теория стала ближе к существующей на тот момент технологии, в частности, в работах Хевисайда, а так же - Лоджа и Герца, давших начало радиосвязи.

Первые упоминания о максвеллианцах стали появляться в 1878, когда Лодж и Фитцежеральд впервые встретились и обнаружили, что оба они полны энтузиазма относительно \emph{Трактата} Максвелла, не смотря на то, что, по их же признанию, у них нет полного понимания всей работы. По словам Фитцжеральда, он изначально предпологал, что теория Максвелла исключает возможность прямой генерации электромагнитных волн, и в 1879 он отговорил Лоджа от попыток их экспериментального возбуждения. Фитцжеральд вскоре осознал свою ошибку и в 1883 публикует работу, в которой описывает методику генерации электромагнитных волн метрового дипазона при помощи разряда конденсатора через малое сопротивление. Но ни сам Фитцежеральд, ни Лодж не смогли придумать способа детектирования столь быстрых колебаний, и, к середине 1880-х они бросили всякие попытки добиться этого.

Позже, в 1883 г., английский ученый Дж. Х. Пойнтинг (J.H.~Poynting) обнаружил, что, согласно теории Максвелла, энергия электромагнитной волны должна распространяться вдоль направления, перпендикулярном как электрическому, так и магнитному полям. Из теоремы Пойнтинга следовало, что энегия электрического тока переносится не по самому кабелю, как было принято думать, а через пространство вокруг него. Фитцжеральд и Лодж вскоре осознали, что теорема Пойнтинга --- ключ к пониманию теории Максвелла, несмотря на то, что сам Максвелл об этом не догадывался. В 1885 г. Фитцжеральд создал механическую модель, состоящую из медных колес и резиновых ремней передачи, для иллюстрации того, как энергия переносится в эфире, в то время как Лодж описал использовал похожую модель для описания передачи энергии через пространство в своей работе \emph{Современные представления об электричестве} (1889 г.).

Хевисайд, экцентричный бывший кабельный инженер, ушедший в ``отставку'' в возрасте 24 лет дабы посвятить себя теории электричества, независимо от Пойнтинга пришел к теореме о плотности потока энергии, которую он так же рассматривал как центральную в теоирии Максвелла. Будучи убежденным, что использованный в оригинальных уравнениях Максвелла вектор потенциал усложняет истинную природу распространения энергии в электромагнитном поле, Хевисайд переписывает уравнения, данные в \emph{Трактате} в компактной форме системы из четырех векторных уравнений, известных сейчас как ``уравнения Максвелла'':
\begin{align*}
&\operatorname{div}\varepsilon{}\mathbf{E} = \rho& 
&\operatorname{rot}\mathbf{H}=\mathrm{k}\mathbf{E} + \varepsilon{}d\mathbf{E}/dt\\
&\operatorname{div}\mu{}\mathbf{H}=0& 
-&\operatorname{rot}\mathbf{E}=\mu{}d\mathbf{H}/dt
\end{align*}
эти уравнения ведут к известному выражению для плотности потока энергии ($\mathbf{S}=\mathbf{E}\times\mathbf{H}$) и, в простой и ясной форме, выражают многие другие аспекты теории Максвелла.

В середине 1880-х, Хевисайд применяет расширенную материальными связями систему уравнений для исследования вопроса распространения электромагнитных волн вдоль электрической линнн --- основную задачу телеграфии. Согласно выводам Хевисайда, сигнал распространяется не внутри провода, но скользит вдоль него в окружающнм пространстве. В 1886 г. он аналитически показывает, что затухание сигнала может быть значительно уменьшено, даже --- устранено полностью, если нагрузить линию специльно подобранной катушкой индуктивности. Метод индуктивной нагрузки, в дальнейшем, показал свою высокую эффективность в телефонной и кабельной индустрии, но сразу же после публикации своего изобретения, Хевисайд встретил жесткий отпор со стороны бывшего в то время главы Британской Почтовой службы (British Post Office), Вильяьма Приса (W.H.~Preece). Прис считал всякую индуктивность в телеграфной линии вредной для распространяющегося сигнала (что, конечно же, верно в случае неверно подобранного значения нагрузочной индуктивности) и предпринял успешные попытки заблокировать публикацию противоречущей его взглядам работы Хевисайда. Прочие важные работы Хевисайда по вопросам теории Максвелла и распространения элеткромагнитных волн, не успевшие привлечь к себе внимания, стали так же преследоваться Присом.

Для Хевисайда было большо удачей, что, начиная с 1888 г., экспериментальные исследования Лоджа в Англии и Герца в Германии, стали привлекать все больше внимания со стороны научного сообщества. Пытаясь получить в своей лаборатории молниеподобные искры посредством быстрой разрядки больших конденсаторов через проволочные перемычки, Лодж обнаружил, что возбуждаются электромагнитные волны, которые скользят вдоль перемычек в окружающем их пространтстве, как то предсказывала теория телеграфа Хевисайда. Вскоре после этого открытия, Лодж и Хевисайд вступили в переписку и стали вместе работать над разработкой теории. Затем, в середине 1888 г., из Германии пришли новости о еще более впечатляющих экспериментах Герца с электромагнитными волнами в воздухе.  В Берлине Герц учился у Гемгольца, от которого узнал о модифицированной версии уравнений Максвелла, и, вдохновившись ими, решил испытать их против дальнодействующих теорий Вебера и Ньюмана. Умея получать короткие электрические вспышки при помощи разряда конденсаторов через воздушные зазоры, Герц сумел зарегестрировать интерференцию возбуждаемых таким образом электромагнитных волн метрового диапазона в  лекционном зале Карлсруэ (Karlsruhe) и измерить их основные параметры.

Экспенрименты Герца были тепло встречены британскими максвеллианцами, которые рассматривали эти результаты как долгожданное подтверждение собственных теоритических предсказаний. В сентябре 1888 г., на встрече Британской Ассоциации, Фитцжеральд заявил, что ``великолепные'' эксперименты Герца окончательно доказали, что электромагнитные cилы не являются дальнодействующими. Лодж, который вместе с Хевисайдом, был близок к открытию, присоединился к поздравлениям, заметив, с иронией, что столь неопревержимое доказательство близкодействующей теории поля Максвелла пришло из Германии --- великой родины дальнодействующих теорий. Континентальные физики вскоре подхватили теорию Максвелла, или, по-крайней мере, модифицированную версию ее уравнений, близких к уравнениям Хевисайда, опубликованных Герцом в 1890 г. Окрепшая теория Максвелла вошла в 1890 г. готовой поглотить в себя не только оптику, но и многие другие направления в физике. По словам Лоджа, электричество стало ``имперской наукой''.

Электричество было имперской наукой в прямом смысле этого слова: подводные кабели, опоясовавшие практически весь мир, были ``нервной системой'' Британской империи. Замечателен тот факт, в свете тесной связи Британской кабельной индустрии и максвеллианской теории, последняя послужила началом новой технологии --- радиосвязи --- которая, в скором времени, разрушила монополию Британии на глобальные телекоммуникации. Герц был сфокусирован на исследовании свойств электромагнитных волн и не рассматривал их как возможное средство сообщения.  Но в 1894 г., Лодж, без всякого на то намерения, случайно изобрел метод радиосообщения при помощи азбуки Морзе. Вслед за ним и независимо друг от друга, Маркони в Италии и Попову в России, удалось собрать первые образцы радипередатчиков, способных применяться на практике. Безпроводной телеграф стал широко использоваться после 1900 г., дополняя и, даже, конкурируя с кабельной сетью. Изоберетение передатчиков непрерывной волны и вакуумных усилительных трубок преобразило индустрию радиосвязи, и первые эксперименты по широковещанию в 1920-х годах перенесли взаимоотношения науки об электричестве и технологии на более высокий уровень.
\section*{Электроны, эфир и относительность}
Прежде чем завоевать свое лидирущее положение в 1890, максвелловская теория была значительно модифицирована. В последующие годы она изменилась еще более значительно. Несмотря на то, что 
\end{document}
