\documentclass[12pt, oneside, a4paper]{article}
\usepackage{ifpdf}
\usepackage[colorlinks,bookmarksopen]{hyperref}
\usepackage[utf8]{inputenc}
\usepackage[english,russian]{babel}
\usepackage{amsmath}
\usepackage{booktabs}
\usepackage{array}
\everymath{\displaystyle}
\begin{document}
\section*{Введение}
В 1864 г. Максвелл, на основе его известных уравнений, предположил, что свет представляет собой поперечные электромагнитные волны. Несмотря на это, сам Максвелл не рассматривал возможность получить свет электромагнитными методами. Впрочем, ничего конкретного, о самих электромагнитных волнах, их генерации и детектировании, в его работах нет. Только спустя почти четверть века с момента опубликования электромагнитной теории света Максвелла, Генрих Герц (Heinrich Hertz) в своих выдающихся  экспериментах зарегестрировал и детально изучил основные свойтсва электромагнитных волн, подтвердив, тем самым, теорию Максвелла. Идеи и уравнений  Максвелла были детально разработаны, модифицированы и облачены в удобную для понимания их физической сути форму усилиями Фитцжеральда (G.F.~FitzGerald), Оливера Лоджа (Oliver Lodge), Оливера Хевисайда (Oliver Heaviside) и Генриха Герца. Первые три ученых образовывали группу, неформально названную Максвеллианцами. 

Данная работа посвящена описанию становления электромагнитной теории света Максвелла как одной из самых ярких обобщающих физических теорий.
\section*{Домаксвелловскыи теории электромагентизма}
Многие натурфилософы и ученые задолго до Максвелла пытались построить теории, объяснящие, каким образом эелектрические и магнитные  силы распространяется через пространство. Великий Карл Фрейдрих Гаусс, ``принц математики'',  в 1855 выдвинул идею, что электрические силы распрстраняются между зарядами с конечной скоростью, но он не решился опубликовать свою работу, потому что он не смог описать мехнизм этого распространения. Неоднократные попытки развить идеи Гаусса были предприняты его учеником Риманом, который в 1853 предложил заменить урвнения Пуассона для электростатического потенциала волновым уравнением, согласно которому, изменения потенциала вследствие изменения заряда должны распространяться в простанстве со скоростью света. Несмотря на то, что это предположение согласуется с современной теорией электромагнетизма, гипотеза Римана была слишком простой, чтобы рассматриваться как основа полноценной теории. В 1867 в \emph{Анналах} Поггендорфа повились две работы. Одна, так и не опубликованная, принадлежала перу Бернхарда Римана, который в 1858 показал, что из модифицированного уравнения Лапласа можно получить волновое решение. Во второй работе Лоренц показал, что в теории Вебера, периодические возмущения электрического заряда способны распространяться со скоротью света.

Известно, что в депозите трудов Королевксого общества (The Royal Society)  ``The Original Views'' есть работа Майкла Фарадея, в которой он выступал с идеей, что эффекты, связанные с электричеством и магнетизмом ``не проявляются мгновенно и распространяются с конечной скоростью''. Фарадей не нашел времени, чтобыв предосвтаить экспериментальные доказательства своей гипотезе, и поэтому пожелал сохранить ее в депозите в 1832 с завещанеим не публиковать раньше чем через 100 лет.
\section*{Максвелл}
Джеймс Клерк Максвелл (James Clerk Maxwell, 1831-1879) был одним из выдающихся физиков девятнадцатого столетия. Спектр проблем, охватываемый его работами, очень широк и включает в себя как кинетическую теори газов, так и теорию цветового восприятия, но наибольшую известность принесла Максвеллу его электромагнитная теория света. Несмотря на то, собрания сочинения Максвелла и исследования его работ несколько раз переиздавались, большинство бумаг, относящиеся непосредственно к личной жизни Максвелла были утеряны после его смерти. Наиболее полная изданная на данный момент биография Максвелла относится к 1882 году.

Максвелл родился и вырос в Эдинбурге, в семье, происходившей из юго-западной Шотландии. После окончания Эдинбургского университета, Максвелл поступил в Кембридж, который закончил в январе 1854 с отличием в математике. В поисках темы для своей исследовательской работы, в следующем месяце он написал он послал письмо Томсону (Thomson), в котором говорил о своем интересе к электричеству и просил рекомендации относительно работ Фарадея. Неизвестно, почему Максвелл выбрал электричсетсво, тему, которая была исключена из учебного курса Кемриджа, и почему он примкнул к непризнанному подходу Фарадея, но стоит заметить, что  его письмо Томсону появилось на волне интереса, поднятого лекцией Фарадея о затухании сигнала в кабеле, данной в Королевском Институте (Royal Institution). 

Первым результатом исследований Максвелла была работа ``О фарадевских силовых линиях'', законченная в начале 1856. Проводя аналогию между силовыми линиями электрического и магнитного полей и потоками в жидкостях, Максвелл облек расплывчатые формулировки Фарадея в строгую математическую форму. В 1861 он выступил с амбициозной работой ``О физических силовых линиях'', основанной на разработанной механической модели эфира, составленного из крошечных вихрей и шестеренок. Эта модель привела Максвелла не только к вравнению, связывающему основыне электромагнитные величины и пониманию, что переменные электрические силы генерируют токи смещения, но также к удивительному выводу, что \emph{свет состоит в поперечных колебаниях той же среды, которая является причиной электрических и магнитных явлений}. 

Электромагнитная теория света по праву считается одной из величайших обобщающих теорий во всей физике, поэтому путь, котоым Максвелл пришел к ней, детально изучен. Единственный вопрос, до сих пор вызывавший споры среди историков и философов науки --- это вопрос о том, насколько реальной Максвелл считал свою модель, изложенную в ``Физических линиях''. Общим мнением счиатетя, что Максвелл действительно расмматривал вихри как реально сущесвтующие в магнитном поле, однако промежуточные шестерни были введены им не более чем для удобства и нагляности теоритеского рассмотрения. 

Лучшим доказетльсвом электромагнитной теории Максвелла было совпадение между скоростью света, измеренным Физо (Hippolyte Fizeau) и другими, и отношением между элеткрическими и магнитными системами единиц, измеренным Вебером (Weber) и Рудольфом Колраушем (Rudolph Kohlraush). Более точное измерение отношения единиц должно было стать серьезным испытанием для его теории, и, надежде на это, Максвелл вступает в Британскую Ассоциацию Комитетов по Электрическом Стандартам (British Committee on Electrical Standarts) в 1862. Затем, будучи профессором в Королевском Лондоском Колледже (Kings College London), Максвелл в течении последующих двух лет тесно работает с инженером-кабельщиком Флемингом Дженклином (Fleeming Jenklin) над вычислением этой константы, используя установку, разработанную Томсоном. Максвелл вычисляет искомое соотношение, и, несмотря на некоторые несостыковки, заключает, что оно находится соответсвии со скоростью света, достаточным для того, чтобы убедиться в правоте собственной теории, несмотря на то, что Томсон не принял это доказательство как решающее. 

В декабре 1864 Максвелл представил Королевскому обществу (Royal Society) свою ``Динамическую теорию электромагнитного поля'', в которой он выводит электромагнитные уравнения не используя какую-либо механическую модель, как "Физическиз линиях", но из общей динамики согласованной системы. Основаясь на фарадеевсих представлениях о зарядах и токах как о побочных свидетельствах состояния окружающих систему полей, Максвелл выражает это состояние через изменение того, что он называет ``электромагнитным моментом'' (позже переименнованым в вектор-потенциал), и описывает, как энергия распределена в полях. Максвелл по-прежнему верил в существование эфира, но, до тех пор, пока до конца не выяснены детали его внутреннего устройства, Максвелл считал, что лучше формулировать теорию с минимумом предполежений. 

Максвелл покинул Королевский Колледж в 1856 и провел следующие несколько лет в своем поместье в Шотландии за написанием  \emph{Трактата об электричестве и магнетизме} (1873). Несмторя на что, оно было полно ценными идеями, в целом, труд получился бессвяным и сложным для понимания. К твоему времени, как труд был опубликован, Максвелл вступил в кавендишское профессорство в области экспериментальной физики в Кембридже, основанном в 1871, и занимался организацией новой лаборатории, основной деятельностью которой он рассматривал проведение точных электрических измерений. Максвелл умер от рака 5 ноября 1879, в возрасте 48 лет, во время работы над переизданием своего Трактата. К сожалению, к этому времени, его теория электромагнитного поля не была ни полностью понятой, ни широко распространненой, более того, в некотором смысле, она не была законченной.
\section*{Теория Максвелла}
В своем \emph{Трактате об электричестве и магнетизме}, Максвелл разработал механическую модель, эквивалентную электрормагнитному полю, но уже в 1864, в работе ``О динамической теории электромагнитного поля'' Максвелл пересматривает свой подход к задаче и, получив сначала законы индукции, выводит из них механгическую модель. В этой работе, представленной Королевскому Обществу, Максвелл представляет 20 уравнений, содержащих 20 неизвестных. Система этих уравнений в математической форме выражает все, что было известно на тот момент об электричестве и магнетизме. В своих уравнениях Максвелл суммирует труда Эрстеда (1777-1851), Карла Гаусса (1777-1855), Ампера (1775-1836), Фарадея (1791-1867) и других, а так же добавляет фундаментальное понятие ``тока смещения''.

\begin{tabular}{>{\raggedright}m{4cm}>{\centering}m{3.5cm}>{\centering}m{3.5cm}}
\toprule
Название физической величины & Обозначения Максвелла & Современная нотация\tabularnewline
\midrule
Вектор потенциал & $F,\,G,\,H$ & $\mathbf{A}$\tabularnewline\tabularnewline
Напряженность магнитного поля & $\alpha{},\,\beta{},\,\gamma{}$ & $\mathbf{H}$\tabularnewline 
\tabularnewline
Напряженность электрического поля & $P,\,Q,\,R$ & $\mathbf{E}$\tabularnewline
\tabularnewline
Ток проводимости & $p,\,q,\,r$ & $\mathbf{J}$\tabularnewline
\tabularnewline
Ток смещения & $f,\,g,\,h$ & $\mathbf{D}$\tabularnewline
\tabularnewline
Полный ток & 
\[
\begin{Bmatrix} 
p^\mathrm{l}=p+\frac{\mathrm{d}f}{\mathrm{d}t}\\
\\
q^\mathrm{l}=q+\frac{\mathrm{d}g}{\mathrm{d}t}\\
\\
r^\mathrm{l}=r+\frac{\mathrm{d}h}{\mathrm{d}t}
\end{Bmatrix}
\]
& $\mathbf{J_T}$\tabularnewline
\tabularnewline
Плотность электрического заряда & $e$ & $\rho$\tabularnewline
\tabularnewline
Электрический потенциал & $\psi$ & $\psi$\tabularnewline
\bottomrule
\end{tabular}

Максвелл вывел свои уравнения для компонент вектора
\section*{Технический прогресс и теория Максвелла}
В вводном слове к своему \emph{Трактату об электричестве и магнетизме}, Максвелл отмечает, что 
\begin{quote}
\small
всевозможные приложения, которые нашла теория электромагнетизма в телеграфии, в конечном итоге, повлияли на развитие чистой науки в том смысле, что составили коммерческую ценность точным электрическим измерениям и предоставили заинтересованным ученым техническое оснащение в масштабах и количестве, недостижимом для рядовой лаборатории. Подобная заинтересованность в знаниях об электричестве и, одновременно, экспериментальная возможность получить их, уже дали свои первые плоды, как в стимулировании усилий видных ученых, так и повсеместном распространении точного знания среди практиков, что, в целом, способствовало  всеобщему научному прогрессу во всей инженерной профессии.
\end{quote}
Максвелл, без сомнений, подразумевал здесь свою работу в Комитете по Стандартам и, конечно же, участие Томсона в работе над проектом прокладки второго Атлантического кабеля, полностью законченную в 1866.

Спустя несколько лет после провала в 1858, Филд и партнеры предприняли вторую попытку проложить кабель по дну Атлантического океана. Следуя рекомендациям, данным Томсоном, уже более последовательно в этот раз, партнеры заказали более тонкий кабель и тщательно его проверили. В июле 1865 корабль \emph{Great Eastern} --- единственный на тот момент во всем флоте, способный нести такое количество кабеля, стал прокладывать его от Ирландии, но, к сожалению, кабель лопнул, когда было проложено около 1200 миль, а его конец был утерян. Проект казался обреченным, но, при поддержке Джона Пендера, богатого торговца хлопком из Манчестера, Филд заказал еще один кабель и предприняли новую попытке следующим летом. В этот раз все прошло гладко и 27 июля 1866 кабель был закреплен вблизи Ньюфаундленда. Вскоре после этого, \emph{Great Eastern} смог подцепить утеряный конец кабеля, оставшийся от неудачной попытки 1865 года, срастил его и закончил прокладку, и, таким образом, с сентября 1866 Атлантика была стянута двумя рабочими кабелями.

Успех 1866 года вызвал всеобщий бум в прокладке кабеля. К 1875, британсике фирмы (владельцем большинстваи из них был тот самый Пендер), проложили кабели к Индии, Австралии, Конг-Гонгу; к 1890 кабели опоясали побережья Южной Америки и Африки, и обеспечивали связью уже весь мир. Телеграфия стала огромной и прибыльной индустрией, сильно заинтересованной в квалифицированных электриках в период с 1850 по 1880. Британские фирмы владели на тот момент примерно двумя третями от общей длины проложенного кабеля, что способствовало повышенному вниманию британских ученых к проведению точных измерений и изучению вопросов электромагнитного распространения.

В конце 1870-х, стали появляться первые электрические станции, ознаменовав тем самым рождение электротехнической промышленности, так же предъявлявшей спрос на знания в области эдектричества. Токи, получаемые от батерей Вольта, или от магнето на постоянных магнитах, были слишком слабы и дорогостоящи для широкого использования. В 1866-1867 несколько изобретателей предложили использовать в магнето электромагниты, что существенно увеличило их мощность и сделало электричество, на первое время, относительно дешевым и доступным. Первым основным применением элетричества, получаемого от динамо-машин, были осветительные дуги, но скоро они сталли слишком яркими для использования их в помещениях. Томас Эдисон (Thomas Edison, 1816-1890), изобревший лампу накаливания в 1879, приступил в разработке системы производства и распределения электрической мощности; в 1882 он установил первые электрические станции на Pearl Street в Нью-Йорке.

Система Эдисона работала на постоянном токе (DC), который подходил для передачи только на короткие расстояния вседствие сильных потерь в линии, поэтому эта система могла обеспечивать электричеством только близлежащих потребителей. Системы переменного тока (AC) обходили это ограничение, использую повышащие трансформаторы перед его передачей на дальние расстояния, и понижающие --- для окончательного распределния между местными потребителями. После нашумевшей ``битвы систем'', переменный ток вышел победителем в 1890.

Ранние DC системы были примитивными, электриченский ток в их проводах мог, в большинстве приложений, рассматриваться как поток воды в трубопроводе. AC системы бли более сложными, в частности, после того как был изобретен метод полифазной передачи, в которых наблюдались многие эффекты, связанные с переменными полями. Стремительный рост электротехнической промышленности в 1880-х и 1890-х годах способствовал безпрецендентному спросу в квалифицированных инженер-электриках, компетентых для работы с  переменным током. Этот спрос нашел отклик в физических факультетах университетов, которые открыли новые лаборатоии и кафедры, чтобы справиться с нахлынувшим потоком студентов. Можно с уверенностью сказать, что развитие физики как дисциплины во второй половине девятнадцатого столтия во многом свзяан со спросом и возможностями, предостваляемыми электрической промышленностью. 
\section*{Максвеллианцы}
Ко времени смерти Максвелла в 1879, его теория электромагнетизма была всего лишь одной из многих других и никоим образом не претендовала на лидерство. Но уже к 1890 она смела своих конкурентов и по-праву заняла место одной из самых удачных и фундаментальных теорий во всей физике. Это стало возможным во многом благодаря работе Генриха Герца и группы ``Максвеллианцев'', среди которых особо выделялись Фитцжеральд, Лодж и Хевисайд. Эти ученые на протяжении 1880-х переформулировали теорию в более четком и компактном виде, подвергли ее экспериментальной проверке  и обобщили на случаи, которые сам Макселл не смог предвидеть. В ходе этой рабоыт, теория стала ближе к существующей на тот момент электротехнике, в частности, в работах Хевисайда, а так же - Лоджа и Герца, давших начало радиосвязи.

Первые упоминания о максвеллианцах стали появляться в 1878, когда Лодж и Фитцежеральд впервые встретились и обнаружили, что оба они полны энтузиазма относительно \emph{Трактата} Максвелла, не смотря на то, что, по их же признанию, у них нет полного понимания всей работы. По словам Фитцжеральда, он изначально предпологал, что теория Максвелла исключает возможность прямой генерации электромагнитных волн, и в 1879 он отговорил Лоджа от попыток их экспериментального возбуждения. Фитцжеральд вскоре осознал свою ошибку и в 1883 опубликовал работу, в которой описывал методику генерации метрового дипазона длин волн при помощи разряда конденсатора через малое сопротивление. Но ни сам Фитцежеральд, ни Лодж не смогли придумать способа детектирования столь быстрых колебаний, и, к середине 1880-х они бросили всякие попытки добиться этого.

Позже, в 1883, английский ученый Дж. Х. Пойнтинг (J.H.~Poynting) обнаружил, что, согласно теории Максвелла, энергия волны должна распространяться вдоль направления, перпендикурного как электрическому, так и магнитному полям. Из теоремы Пойнтинга следовало, что энегия электрического тока переносится не по самому кабелю, как было принято думать, а перенесится через пространстов вокруг него. Фитцжеральд и Лодж вскоре пришли к пониманию, что теорема Пойнтинга --- ключ к пониманию теории Максвелла, несмотря на то, что сам Максвелл об этом не догадывался. В 1885 создал механическую модель, состоящую из медных колес и резиновых ремней передачи, для иллюстрации того, как энергия переносится в эфире, в то время как Лодж описал работу похожей модели на расстоянии в своей раьоте \emph{Современные представления об электричестве} (1889).

Хевисайд, экцентричный бывший кабельный инженер, ушедший на в ``отставку'' в возрасте 24 лет дабы посвятить себя теории электричества, независимо от Пойнтинга пришел к теореме о потоке энергии, которую он так же рассматривал как центральную в теоии Максвелла. Будучи убежденным, что исользованный в оригинальных уравнениях Максвелла вектор-потенциал скрывает истинную природу распространения энергие в электромагнитном поле, Хевисайд переписал уравнения, данные в \emph{Трактате} в компактной форме системы из четырех вектроных уравнений, известной сейчас как ``уравнения Максвелла'':
\begin{align*}
&\operatorname{div}\varepsilon{}\mathbf{E} = \rho& 
&\operatorname{curl}\mathbf{H}=\mathrm{k}\mathbf{E} + \varepsilon{}d\mathbf{E}/dt\\
&\operatorname{div}\mu{}\mathbf{H}=0& 
-&\operatorname{curl}\mathbf{E}=\mu{}d\mathbf{H}/dt
\end{align*}
эти уравнения ведут к известному выражению для плотности потока энергии ($\mathbf{S}=\mathbf{E}\times\mathbf{H}$) и, в простой и ясной форме, выражают многие другие аспекты теории Максвелла.

В середине 1880-х, Хевисайд применил расширенную версию своих уравнений совместно с маетриальными уравнениями для исследования вопроса распространения электромагнитных волн вдоль проводов --- основную задачу телеграфии. Согласно выводам Хевисайда, сигнал распространяетс не вунутри провода, но скользит вдоль него в окружающнм пространстве. В 1886 он аналитически показал, что затухание сигнала может быть значительно уменьшено, даже --- устранено полностью, если нагрузить линию специльно подобранным индуктивным сопротивлением. Метод индуктивной нагрузки в дальнейшем показал высокую эффективность в телефонной и кабельной индустрии, но сразу же после публикации своего изобретения, Хевисайд встретил жесткий отпор со стороны тогдашнего главы Британской Почтовой службы (British Post Office), Вильяьма Приса (W.H.~Preece). Прис считал всякую индуктивность в линии вредной для распространяющегося сигнала (что, конечно же, верно в случае неверно подобранного значения нагрузочной индуктивности) и предпринял попытки заблокировать публикацию противоречащего его взглядам работы Хевисайда. Прочие важные работы Хевисайда в области по вопросам теории Максвелла и распространения элеткромагнитных волн, не успевшие привлечь к себе внимания, стали так же преследоваться Присом.


Для Хевисайда было большо удачей, что, начиная с 1888, экспериментальные исследования Лоджа в Англии и Герца в Германии, стали привлекать все больше внимания со стороны научного сообщества. Пытаясь имитировать молнии  в лаборатории посредством разряда больших конденсаторов через перемычки, Лодж обнаружил, что электромагнитные волны скользят вдоль перемычек в окружающем их пространтстве, как то предсказывала теория телеграфа Хевисайда. Вскоер после этого Лодж и Хевисайд вступили в переписку и стали близкими компаньонами. Затем, в середине 1888, из Германии пришли новости о еще более впечатляющих экспериментах Герца с электромагнитными волнами в воздухе.  В Берлине Герц учился у Гемгольца, от которго узнал о модифицированнйо версии уравнений Максвелла, и, вдохновившись ими, решил испытать их против дальнодействующих теорий Вебера и Ньюмана. Умея воспроизводить короткие электрические вспышки при помощи быстрых колебаний электричесого тока, Герц сумел добиться интерференции электромагнитных волн метрового диапазона длин лекциооннм зале Карлсруэ (Karlsruhe) и измерить их основные параметры.

Экспенрименты Герца были тепло встречены британсикими максвеллианцами, которые рассматривали их результаты как долгожданное потверждение собственных теоритических предсказаний. В сентябре 1888, на встрече Британской Ассоциации, Фитцжеральд заявил, что ``великолепные'' эксперименты Герца окончательно доказали, что электромагнитные cилы не являются дальнодействующими. Лодж, который вместе с Хевисайдом, был близок к открытию, присоединился к поздравлениям, заметив, с иронией, что столь неопревержимое доказательство теории поля Максвелла пришло из Германии --- великой родины дальнодействующих теорий. Континентальные физики вскоре подхватили теорию Максвелла, или, по-крайней мере, упрощенную версию ее уравнений, близких к уравнениям Хевисайда, опубликованных Максвеллом в 1890. Окрепшая теория Максвелла вошла в 1890 готовой поглотить не толкьо оптику, но и многие другие направления в физике. По словам Лоджа, электричество стало ``имперской наукой''.

Электричество было имперской наукой в прямом смысле этого слова: подводные кабели, опоясовавшие практически весь мир, были ``нервной системой'' Британской империи. Замечателен тот факт, в свете тесной связи Британской кабельной индустрии и Максвеллианской теории, последняя послужила началом новой технологии --- радиосвязи --- которая, в скором времени, разрушила монополию Британии на глобальные телекоммуникации. Герц был сфокусирован на исследовании свойств электромагнитных волн и не рассматривал их как средсво сообщения.  Но в 1894, Лодж, без всякого на то намерения, случайно изобрел метод радиосообщения при помощи азбуки0 Морзе. Вслед за ним и независимо друг от друга, Маркони в Италии и Попову в России, удалось собрать первые образцы радипередатчиков, способных применяться на практике. Безпроводной телеграф стал широко использоваться после 1900, дополняя и конкурируя с кабельной сетью.Изоберетение передатчиков непрерывной волны и вакуумных усилительных трубок преобразило индустрию радиосвязи, и первые эксперименты по широковещанию в 1920-х перенесли взаимоотношения науки об электричестве и технологии на более высокий уровень.
\section*{Электроны, эфир и относительность}
Прежде чем завоевать свое лидирущее положение в 1890, максвелловская теория была значительно модифицирована. В последующие годы она изменилась еще более значительно. Несмотря на то, что 
\end{document}
