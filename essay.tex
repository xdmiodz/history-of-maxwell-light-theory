\documentclass[12pt, oneside, a4paper]{article}
\usepackage{ifpdf}
\usepackage[colorlinks,bookmarksopen]{hyperref}
\usepackage[utf8]{inputenc}
\usepackage[english,russian]{babel}
\begin{document}
\section*{Введение}
В 1864 г. Максвелл, на основе его известных уравнений, предположил, что свет представляет собой поперечные электромагнитные волны. Несмотря на это, сам Максвелл не рассматривал возможность получить свет электромагнитными методами. Впрочем, ничего конкретного, о самих электромагнитных волнах, их генерации и детектировании, в его работах нет. Только спустя почти четверть века с момента опубликования электромагнитной теории света Максвелла, Генрих Герц (Heinrich Hertz) в своих выдающихся  экспериментах зарегестрировал и детально изучил основные свойтсва электромагнитных волн, подтвердив, тем самым, теорию Максвелла. Идеи и уравнений  Максвелла были детально разработаны, модифицированы и облачены в удобную для понимания их физической сути форму усилиями Фитцжеральда (G.F.~FitzGerald), Оливера Лоджа (Oliver Lodge), Оливера Хевисайда (Oliver Heaviside) и Генриха Герца. Первые три ученых образовывали группу, неформально названную Максвеллианцами. 

Данная работа посвящена описанию становления электромагнитной теории света Максвелла как одной из самых ярких обобщающих физических теорий.

\section*{Максвелл}
Джеймс Клерк Максвелл (James Clerk Maxwell, 1831-1879) был одним из выдающихся физиков девятнадцатого столетия. Спектр проблем, охватываемый его работами, очень широк и включает в себя как кинетическую теори газов, так и теорию цветового восприятия, но наибольшую известность принесла Максвеллу его электромагнитная теория света. Несмотря на то, собрания сочинения Максвелла и исследования его работ несколько раз переиздавались, большинство бумаг, относящиеся непосредственно к личной жизни Максвелла были утеряны после его смерти. Наиболее полная изданная на данный момент биография Максвелла относится к 1882 году.

Максвелл родился и вырос в Эдинбурге, в семье, происходившей из юго-западной Шотландии. После окончания Эдинбургского университета, Максвелл поступил в Кембридж, который закончил в январе 1854 с отличием в математике. В поисках темы для своей исследовательской работы, в следующем месяце он написал он послал письмо Томсону (Thomson), в котором говорил о своем интересе к электричеству и просил рекомендации относительно работ Фарадея. Неизвестно, почему Максвелл выбрал электричсетсво, тему, которая была исключена из учебного курса Кемриджа, и почему он примкнул к непризнанному подходу Фарадея, но стоит заметить, что  его письмо Томсону появилось на волне интереса, поднятого лекцией Фарадея о задержках в кабелях (cable retardation), данной в Королевском Институте (Royal Institution). 

Первым результатом исследований Максвелла была работа "О фарадевских силовых линиях", законченная в начале 1856. Проводя аналогию между силовыми линиями электрического и магнитного полей и потоками в жидкостях, Максвелл облек расплывчатые формулировки Фарадея в строгую математическую форму. В 1861 он выступил с амбициозной работой "О физических силовых линиях", основанной на разработанной механической модели эфира, составленного из крошечных вихрей и шестеренок. Эта модель привела Максвелла не только к вравнению, связывающему основыне электромагнитные величины и пониманию, что переменные электрические силы генерируют токи смещения, но также к удивительному выводу, что \emph{свет состоит в поперечных колебаниях той же среды, которая является причиной электрических и магнитных явлений}. 

Электромагнитная теория света по праву считается одной из величайших обобщающих теорий во всей физике, поэтому путь, котоым Максвелл пришел к ней, детально изучен. Единственный вопрос, до сих пор вызывавший споры среди историков и философов науки - это вопрос о том, насколько реальной Максвелл считал свою модель, изложенную в "Физических линиях". Общим мнением счиатетя, что Максвелл действительно расмматривал вихри как реально сущесвтующие в магнитном поле, однако промежуточные шестерни были введены им не более чем для удобства и нагляности теоритеского рассмотрения. 

Лучшим доказетльсвом электромагнитной теории Максвелла было совпадение между скоростью света, измеренным Физо (Hippolyte Fizeau) и другими, и отношением между элеткрическими и магнитными системами единиц, измеренным Вебером (Weber) и Рудольфом Колраушем (Rudolph Kohlraush). Более точное измерение отношения единиц должно было стать серьезным испытанием для его теории, и, надежде на это, Максвелл вступает в Британскую Ассоциацию Комитетов по Электрическом Стандартам (British Committee on Electrical Standarts) в 1862. Затем, будучи профессором в Королевском Лондоском Колледже (Kings College London), Максвелл в течении последующих двух лет тесно работает с инженером-кабельщиком Флемингом Дженклином (Fleeming Jenklin) над вычислением этой константы, используя установку, разработанную Томсоном. Максвелл вычисляет искомое соотношение, и, несмотря на некоторые несостыковки, заключает, что оно находится соответсвии со скоростью света, достаточным для того, чтобы убедиться в правоте собственной теории, несмотря на то, что Томсон не принял это доказательство как решающее. 

В декабре 1864 Максвелл представил Королевскому обществу (Royal Society) свою "Динамическую теорию электромагнитного поля", в которой он выводит электромагнитные уравнения не используя какую-либо механическую модель, как "Физическиз линиях", но из общей динамики согласованной системы. Основаясь на фарадеевсих представлениях о зарядах и токах как о побочных свидетельствах состояния окружающих систему полей, Максвелл выражает это состояние через изменение того, что он называет "электромагнитным моментом" (позже переименнованым в вектор-потенциал), и описывает, как энергия распределена в полях. Максвелл по-прежнему верил в существование эфира, но, до тех пор, пока до конца не выяснены детали его внутреннего устройства, Максвелл считал, что лучше формулировать теорию с минимумом предполежений. 

Максвелл покинул Королевский Колледж в 1856 и провел следующие несколько лет в своем поместье в Шотландии за написанием  \emph{Трактата об электричестве и магнетизме} (1873). Несмторя на что, оно было полно ценными идеями, в целом, труд получился бессвяным и сложным для понимания. К твоему времени, как труд был опубликован, Максвелл вступил в кавендишское профессорство в области экспериментальной физики в Кембридже, основанном в 1871, и занимался организацией новой лаборатории, основной деятельностью которой он рассматривал проведение точных электрических измерений. Максвелл умер от рака 5 ноября 1879, в возрасте 48 лет, во время работы над переизданием своего Трактата. К сожалению, к этому времени, его теория электромагнитного поля не была ни полностью понятой, ни широко распространненой, более того, в некотором смысле, она не была законченной.

\section*{Технический прогресс и теория Максвелла}
В вводном слове к своему \emph{Трактату об электричестве и магнетизме}, Максвелл отмечает, что 
\begin{quote}
\small
всевозможные приложения, которые нашла теория электромагнетизма в телеграфии, в конечном итоге, повлияли на развитие чистой науки в том смысле, что составили коммерческую ценность точным электрическим измерениям и предоставили заинтересованным ученым техническое оснащение в масштабах и количестве, недостижимом для рядовой лаборатории. Подобная заинтересованность в знаниях об электричестве и, одновременно, экспериментальная возможность получить их, уже дали свои первые плоды, как в стимулировании усилий видных ученых, так и повсеместном распространении точного знания среди практиков, что, в целом, способствовало  всеобщему научному прогрессу во всей инженерной профессии.
\end{quote}
Максвелл, без сомнений, подразумевал здесь свою работу в Комитете по Стандартам и, конечно же, работу Томсона в проекте прокладки второго Атлантического кабеля, полностью законченную в 1866.

Спустя несколько лет после провала в 1858, Филд и партнеры предприняли вторую попытку проложить кабель по дну Атлантического океана. Следуя рекомендациям, данным Томсоном, уже более последовательно в этот раз, партнеры заказали более тонкий кабель и тщательно его проверили. В июле 1865 корабль \emph{Great Eastern} --- единственный на тот момент во всем флоте, способный нести такое количество кабеля, стал прокладывать его от Ирландии, но, к сожалению, кабель лопнул, когда было проложено около 1200 миль, а его конец был утерян. Проект казался обреченным, но, при поддержке Джона Пендера, богатого торговца хлопком из Манчестера, Филд заказал еще один кабель и предприняли новую попытке следующим летом. В этот раз все прошло гладко и 27 июля 1866 кабель был закреплен вблизи Ньюфаундленда. Вскоре после этого, \emph{Great Eastern} смог подцепить утеряный конец кабеля, оставшийся от неудачной попытки 1865 года, срастил его и закончил прокладку, и, таким образом, с сентября 1866 Атлантика была стянута двумя рабочими кабелями.

Успех 1866 года вызвал всеобщий бум в прокладке кабеля. К 1875, британсике фирмы (владельцем большинстваи из них был тот самый Пендер), проложили кабели к Индии, Австралии, Конг-Гонгу; к 1890 кабели опоясали побережья Южной Америки и Африки, и обеспечивали связью уже весь мир. Телеграфия стала огромной и прибыльной индустрией, сильно заинтересованной в квалифицированных электриках в период с 1850 по 1880. Британские фирмы владели на тот момент примерно двумя третями от общей длины проложенного кабеля, что способствовало повышенному вниманию британских ученых к проведению точных измерений и изучению вопросов электромагнитного распространения.

В конце 1870-х, стали появляться первые электрические станции, ознаменовав тем самым рождение электротехнической промышленности, так же предъявлявшей спрос на знания в области эдектричества. Токи, получаемые от батерей Вольта, или от магнето на постоянных магнитах, были слишком слабы и дорогостоящи для широкого использования. В 1866-1867 несколько изобретателей предложили использовать в магнето электромагниты, что существенно увеличило их мощность и сделало электричество, на первое время, относительно дешевым и доступным. Первым основным применением элетричества, получаемого от динамо-машин, были осветительные дуги, но скоро они сталли слишком яркими для использования их в помещениях. Томас Эдисон (Thomas Edison, 1816-1890), изобревший лампу накаливания в 1879, приступил в разработке системы производства и распределения электрической мощности; в 1882 он установил первые электрические станции на Pearl Street в Нью-Йорке.

Система Эдисона работала на постоянном токе (DC), который подходил для передачи только на короткие расстояния вседствие сильных потерь в линии, поэтому эта система могла обеспечивать электричеством только близлежащих потребителей. Системы переменного тока (AC) обходили это ограничение, использую повышащие трансформаторы перед его передачей на дальние расстояния, и понижающие --- для окончательного распределния между местными потребителями. После нашумевшей "битвы систем", переменный ток вышел победителем в 1890.

Ранние DC системы были примитивными, электриченский ток в их проводах мог, в большинстве приложений, рассматриваться как поток воды в трубопроводе. AC системы бли более сложными, в частности, после того как был изобретен метод полифазной передачи, в которых наблюдались многие эффекты, связанные с переменными полями. Стремительный рост электротехнической промышленности в 1880-х и 1890-х годах способствовал безпрецендентному спросу в квалифицированных инженер-электриках, компетентых для работы с  переменным током. Этот спрос нашел отклик в физических факультетах университетов, которые открыли новые лаборатоии и кафедры, чтобы справиться с нахлынувшим потоком студентов. Можно с уверенностью сказать, что развитие физики как дисциплины во второй половине девятнадцатого столтия во многом свзяан со спросом и возможностями, предостваляемыми электрической промышленностью. 
\end{document}
