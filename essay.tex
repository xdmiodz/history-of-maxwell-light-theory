\documentclass[12pt, oneside, a4paper]{article}
\usepackage{pscyr}
\usepackage[colorlinks,bookmarksopen]{hyperref}
\usepackage[warn]{mathtext}
\usepackage[english,russian]{babel}
\usepackage{amsmath,amssymb}
\usepackage{booktabs}
\usepackage{array}
\renewcommand{\rmdefault}{ftm}
\usepackage[utf8]{inputenc}
\DeclareSymbolFont{T2Aletters}{T2A}{cmr}{m}{it}
\everymath{\displaystyle}
\begin{document}
\tableofcontents
\newpage
\section{Введение}
Данная работа содержит краткий очерк об истории создания электромагнитной теории Максвелла и теоретической и экспериментальной работы над ней, как самого Максвелла, там и его последователей. В первой главе рассматриваются ранние теории, пытавшиеся объяснить феномен распространения электромагнитных сил в пространстве. Эти теории, в большинстве своем, рассматривали свет и электромагнетизм как независимые явления, однако, именно они повлияли на исследования Максвелла вопросов физической природы света. Во второй главе дается краткий обзор жизни Максвелла и основных его работ и идей, сформировавшихся, в итоге, в виде знаменитых уравнений. В третьей главе, на основе теории Максвелла, в том виде, в котором он оставил ее после своей смерти, делается попытка объяснить, как Максвелл пришел к своему замечательному выводу об единой физической природе электромагнетизма и света. В четвертой главе делается обзор работы последователей Максвелла, в основном, Генриха Герца в Германии и группы Максвеллианцев в Англии, над теорией электромагнетизма. В пятой главе описывается научно-техническое состояние в мире на момент работы Максвелла и его последователей. Наконец, в шестой главе кратко  рассказывается об истории развития математических методов описания физических законов, что позволило записать теорию Максвелла в виде, позволявшем ее экспериментальную интерпретацию.
\section{Ранние теории электромагнетизма}
Задолго до Максвелла, многие ученые и натурфилософы пытались объяснить, каким образом действие электрических и магнитных  сил распространяется в пространстве. Карл Фрейдрих Гаусс, ``принц математики'', в 1855~г. выдвинул идею об ограниченной скорости распространения электрического взаимодействия между зарядами, но решил не публиковать свою работу, поскольку не смог построить модель в среды, в котором такое распространение было бы возможным. Неоднократные попытки развить идеи Гаусса были предприняты его учеником, Риманом, который в 1853~г., основываясь на уравнении Пуассона для электростатического потенциала, получил волновое уравнение, согласно которому, изменение заряда влечет за собой возмущение потенциала, которое распространяется в пространстве со скоростью света. Несмотря на то, что это предположение согласуется с современной теорией электромагнетизма, гипотеза Римана была тривиальной, чтобы рассматриваться как основа полноценной теории. В 1867~г. в \emph{Анналах} Поггендорфа появились две работы. Одна из этих работ, так и не опубликованная, принадлежала Бернхарду Риману, который в 1858~г. показал, что из обобщенного уравнения Лапласа возможно получить волновое решение. Во второй работе, Лоренц показал, что в теории Вебера, периодические возмущения электрического заряда способны распространяться со скоростью света.

В депозите Королевского общества находится работа, озаглавленная ``The Original Views'', в которой Майкл Фарадей выступает с идеей, что эффекты, связанные с электричеством и магнетизмом ``не проявляются мгновенно и распространяются с конечной скоростью''. Фарадей не нашел времени, чтобы провести экспериментальную проверку своей гипотезы, и поэтому пожелал сохранить свою работу 1832 года в депозите, с завещанием не публиковать ее минимум 100 лет.
\section{Научная жизнь Максвелла}
Джеймс Клерк Максвелл (1831--1879~гг.) был одним из выдающихся физиков девятнадцатого столетия. Спектр проблем, охватываемый его работами, простирается от кинетической теории газов до теории цветового восприятия, но наибольшую известность принесла Максвеллу его электромагнитная теория света. Несмотря на то, что собрания сочинений Максвелла и исследования его работ несколько раз переиздавались, большинство бумаг, относящиеся непосредственно к личной жизни Максвелла, были утеряны после его смерти. Наиболее полная, на данный момент, биография Максвелла была издана в 1882 году.

Максвелл родился и вырос в Эдинбурге, в семье, происходившей из юго-западной Шотландии. После окончания Эдинбургского университета, Максвелл поступает в Кембридж, который заканчивает в январе 1854~г. с отличием в математике. В поисках темы для своей исследовательской работы, в следующем месяце он пишет письмо Томсону, в котором рассказывает о своем интересе к теории электричества и просит рекомендаций относительно работ Фарадея. Неизвестно, почему Максвелл выбрал электричество, тему, которая не была включена в учебный курс Кембриджа, и почему он примкнул к неортодоксальному подходу Фарадея к проблеме распространения электромагнитных сил, но стоит заметить, что  его письмо Томсону появляется на волне интереса, поднятого лекцией Фарадея о затухании электромагнитного сигнала в кабеле, данной в Королевском Институте. 

Первым результатом исследований Максвелла была работа ``О фарадеевских силовых линиях'', законченная в начале 1856~г. Проводя аналогию между силовыми линиями электрического и магнитного полей и потоками в жидкости, Максвелл облек расплывчатые формулировки Фарадея в строгую математическую форму. В 1861 г. он выступил с амбициозной работой ``О физических силовых линиях'', основанной на разработанной им механической модели эфира, состоящего из крошечных вихрей и шестеренок. Эта модель привела Максвелла не только к уравнению, связывающему основные электромагнитные величины и пониманию, что переменные электрические силы производят ток смещения, но также к удивительному выводу, что \emph{свет состоит в поперечных колебаниях той же среды, которая является причиной электрических и магнитных явлений}. 
Электромагнитная теория света по праву считается одной из величайших обобщающих теорий во всей физике, поэтому путь, которым Максвелл пришел к ней, был детально изучен. Единственный вопрос, до сих пор вызывавший споры среди историков и философов науки --- это вопрос о том, насколько реальной Максвелл считал свою механическую модель эфира, изложенную в ``Физических линиях''. Общим мнением считается, что Максвелл действительно рассматривал вихри как реально существующие в магнитном поле, однако, промежуточные шестерни были введены им не более чем для удобства и наглядности теоретического рассмотрения. 

Основным доказательством электромагнитной теории Максвелла было совпадение между скоростью света, измеренной Физо и другими, и отношением между электрическими и магнитными системами единиц, измеренным Вебером  и Рудольфом Колраушем. Более точное измерение отношения единиц было бы серьезным испытанием для теории Максвелла, и, надежде на это, он вступает в Британский Комитет по Электрическом Стандартам в 1862~г. Затем, будучи уже профессором в Королевском Лондоском Колледже, Максвелл в течении последующих двух лет тесно работает с инженером-кабельщиком Флемингом Дженклином над вычислением искомого соотношения, используя методику, разработанную Томсоном. Максвелл вычисляет искомое соотношение, и заключает, что оно находится в соответствии с известным на тот момент значением скорости света, что было достаточно, чтобы убедиться в правоте собственной теории. Стоит заметить, что Томсон не принял этого доказательства как решающее. 

В декабре 1864~г. Максвелл представляет Королевскому Обществу свою ``Динамическую теорию электромагнитного поля'', в которой он выводит уравнения электромагнитных полей, не пользуясь какой либо механической моделью, как это было в ``Физических линиях'', но из общей динамики согласованной системы. Основываясь на фарадеевских представлениях о зарядах и токах как о вторичных свидетельствах состояния окружающих систему полей, Максвелл выражает это состояние через изменение того, что он называет ``электромагнитным моментом'' (позже переименованным в вектор потенциал), и описывает, как распределяется энергия в электромагнитном поле. Максвелл по-прежнему верит в существование эфира, но, до тех пор, пока до конца не выяснены детали его внутреннего устройства, Максвелл считает, что лучше формулировать теорию, используя минимум предположений. 

Максвелл покинул Королевский Колледж в 1856~г. и провел следующие несколько лет в своем поместье в Шотландии за написанием  \emph{Трактата об электричестве и магнетизме} (1873~г.). Несмотря на что, оно было полно ценными идеями, труд, в целом, получился бессвязным и сложным для понимания. К тому времени, как \emph{Трактат} был опубликован, Максвелл вступает в основанное в 1871~г. кавендишское профессорство в области экспериментальной физики, в Кембридже, и занимается организацией новой лаборатории, основной деятельностью которой он рассматривает проведение прецизионных электрических измерений. Максвелл умер от рака 5 ноября 1879~г., в возрасте 48 лет, во время работы над переизданием \emph{Трактата}. К сожалению, к этому времени, его электромагнитная теория света не была ни полностью понятой, ни широко распространенной, более того, её нельзя было считать законченной.
\section{Теория Максвелла}
Впервые, уравнения Максвелла, связывающие воедино, в математической форме, всё известное на том момент об электромагнетизме, появляются в работе 1864~г. ``Динамическая теория электромагнитного поля''. В этой работе Максвелл представляет 20 уравнений, содержащих 20 неизвестных.  В своих уравнениях Максвелл суммирует труды Эрстеда (1777--1851~гг.), Карла Гаусса (1777--1855~гг.), Ампера (1775--1836~гг.), Фарадея (1791--1867~гг.) и многих других, а так же добавляет фундаментальное понятие ``тока смещения''.

Для дальнейшего изложения полезно будет рассмотреть  уравнения Максвелла, в том виде, в котором он изложил их в своей работе. В таблице [\ef{tab:maxwell_vars}] приведены обозначения физических величин, использованных Максвеллом и их современная нотация. 

Наборы из трех величин, представленные в среднем столбце первых шести строчек таблицы [\ref{tab:maxwell_vars}], представляют собой декартовы компоненты ${x,\,y,\,z}$ соответствующих векторов в левом крайнем столбце. Помимо переменных, указанных в таблице, Максвелл неявно пользуется ``магнитной индукцией'', три компоненты вектора которой, в изотропной среде, он обозначает как $\mu_\alpha,\,\mu_\beta,\,\mu_\gamma$. Современное название для этой величины --- вектор напряженности магнитного поля, $\mathbf{B}=\mu\mathbf{H}$, где $\mu$ определяется средой.
\begin{table}[p]
\begin{tabular}{>{\raggedright}m{4cm}>{\centering}m{3.5cm}>{\centering}m{3.5cm}}
\toprule
Название физической величины & Обозначения Максвелла & Современная нотация\tabularnewline
\midrule
Вектор потенциал & $F,\,G,\,H$ & $\mathbf{A}$\tabularnewline\tabularnewline
Индукция магнитного поля & $\alpha{},\,\beta{},\,\gamma{}$ & $\mathbf{H}$\tabularnewline 
\tabularnewline
Напряженность электрического поля & $P,\,Q,\,R$ & $\mathbf{E}$\tabularnewline
\tabularnewline
Ток проводимости & $p,\,q,\,r$ & $\mathbf{J}$\tabularnewline
\tabularnewline
Ток смещения & $f,\,g,\,h$ & $\mathbf{D}$\tabularnewline
\tabularnewline
Полный ток & 
\[
\begin{Bmatrix} 
p^\mathrm{l}=p+\frac{\mathrm{d}f}{\mathrm{d}t}\\
\\
q^\mathrm{l}=q+\frac{\mathrm{d}g}{\mathrm{d}t}\\
\\
r^\mathrm{l}=r+\frac{\mathrm{d}h}{\mathrm{d}t}
\end{Bmatrix}
\]
& $\mathbf{J_T}$\tabularnewline
\tabularnewline
Плотность свободного заряда & $e$ & $\rho$\tabularnewline
\tabularnewline
Электрический потенциал & $\psi$ & $\psi$\tabularnewline
\bottomrule
\end{tabular}
\caption{Оригинальные обозначения, использованные Максвеллом в своих уравнениях}
\label{tab:maxwell_vars}
\end{table}

Используя указанные 20 переменных, приведенных в таблице [\ref{tab:maxwell_vars}], Максвелл, в приближении изотропной среды, выписывает уравнения, формирующие базис ``Динамической теории электромагнитного поля'', записанные здесь, для удобства, в векторной форме с использование современных обозначений.
\begin{align}
&\mathbf{J_T}=\mathbf{J}+\frac{\partial{\mathbf{D}}}{\partial{t}}\label{eq:A}\\
&\mu\mathbf{H} = \mathbf{B} = \nabla\times\mathbf{A}\label{eq:B}\\
&\nabla\times\mathbf{H}=4\pi\mathbf{J_T}=4\pi\left[\mathbf{J}+\frac{\partial{\mathbf{D}}}{\partial{t}}\right]\label{eq:C}\\
&\mathbf{E}=\boldsymbol\nu\times\mathbf{B} - \frac{\partial{\mathbf{D}}}{\partial{t}} - \nabla\psi\label{eq:D}
\end{align}
Уравнение~\eqref{eq:D} названо Максвеллом уравнением для электродвижущей силы в проводнике, движущимся со скоростью $\boldsymbol\nu$ в изотропной среде.
\begin{align}
\mathbf{E}=\mathrm{k}\mathbf{D},\label{eq:E}
\end{align}
где $\mathrm{k}$ назван Максвеллом ``коэффициентом электрической упругости''. Для сравнения, современная запись уравнения~\eqref{eq:E} выглядит как $\mathbf{D}=\varepsilon\mathbf{E}$, $\varepsilon$---диэлектрическая проницаемость среды.
\begin{align}
\mathbf{E}=\rho^{'}\mathbf{J},\label{eq:F}
\end{align}
где $\rho^{'}$ характеризует ``сопротивляемость'' или, иначе, сопротивление среды. Максвелл использовал символ $\rho$ вместо $\rho^{'}$, однако, чтобы не вступать в конфликт с обозначением для объемной плотности заряда, здесь используется $\rho^{'}$. В современной нотации, уравнение~\eqref{eq:F} записывается как $\mathbf{J}=\sigma\mathbf{E}$, где $\sigma=1/\rho^{'}$.
\begin{align}
&\nabla\cdot\mathbf{D}=\rho\label{eq:G}\\
&\nabla\cdot\mathbf{J}+\frac{\partial{\rho}}{\partial{t}}=0\label{eq:H}
\end{align}
Ясно, что векторные уравнения \eqref{eq:A}--\eqref{eq:H} образуют 20 скалярных уравнений. Стоит отметить, что Максвелл использовал смешанную гауссову систему единиц, в которой электрический и магнитные величины выражены в системах СГСЭ и СГСМ, соответственно. Появление множителя $4\pi$ в уравнении~\eqref{eq:C} --- одно из следствий выбора этой системы единиц. Для понимания того, как Максвелл сделал свой знаменитый вывод о единой природе электромагнетизма и света, необходимо использовать ту же самую систему единиц, ту же самую форму записи уравнений электромагнетизма.

Система СГСЭ базируется на определении единичного заряда, который определятся так, что две частицы, имеющие единичный заряд каждая, будучи разнесенными на расстояние единичной длины, испытывают единичную силу взаимодействия. Из этого определения выводятся размерности электрической силы, электрического потенциала, электрического тока и т.д. В СГСМ за основу взят единичный магнитный монополь, который определяется так, что два единичных магнитных монополя, разнесенные на расстояние единичной длины, испытывают единичную силу взаимодействия. Отсюда выводятся размерности магнитной индукции, магнитного потенциала, напряженности магнитного поля и т.д. Максвелл использовал смесь этих систем единиц --- гауссову систему, в которой электрические явления описываются в СГСЭ, а магнитные --- в системе СГСМ.

Использование параметра $\mathrm{k}$ в уравнении~\eqref{eq:E} нуждается в пояснении. Для описания механических сил, испытываемых электрическими и магнитными зарядами  и связи между величинами в СГСЭ и СГСМ, Максвелл показал, что:\begin{align}
\mathrm{k}=4\pi\frac{C^2}{K},\label{eq:5.1}
\end{align}
где $K$--- ``диэлектрическая проницаемость'' среды, а
\[
C=(единичный\,эл.\,заряд\,в\,СГСМ)/(единичный\,эл.\,заряд\,в\,СГСЭ)
\]
в гауссовой системе единиц в свободном пространстве ($\mu=1$, $K=1$).

Если электрические явления были бы никак не связаны с магнитными, тогда не было бы необходимости использовать две совершенно разные системы единиц, и какая либо связь между эти двумя системами отсутствовала бы. Но тот экспериментальный факт, что существует связь между электрическим током и магнитным полем позволяет определить и связь между двумя системами единиц. Так же рассуждал и Максвелл. С этого момента важно заметить, что Максвелл был родоначальником размерного анализа. Он был первым, кто приписал размерности всем величинам в этих двух системах и вывел выражения для тех же величин в этих системах. Максвелл выписал выражения для электрического тока в СГСЭ и СГСМ и получил, что отношение между ними:
\begin{align*}
\frac{эл.\,ток\,СГСЭ}{эл.\,ток\,СГСМ}&=\frac{M^{1/2}L^{3/2}T^{-2}K^{1/2}}{M^{1/2}L^{1/2}T^{-1}\mu^{-1/2}}=\frac{L/T}{\frac{1}{\sqrt{K\mu}}}\\&=\frac{ед.\, скорости}{скорость}=число
\end{align*}
Как оказалось, это отношение равно отношению единичной скорости и некой скорости, т.е. является некой безразмерной константой. Вебер и Колрауш впервые вычислили это отношение, сравнивая результаты измерения емкости конденсатора двумя способами: электростатически, сравнивая неизвестную емкость с емкостью сферы известного радиуса, и электромагнитными измерениями, разряжая конденсатор через гальванометр. Результаты этих измерений свидетельствовали о том, что искомое соотношение равно, приблизительно, $3.0001\times{}10^{10}$\,см/с. Для скорости распространения света в вакууме ранее было получено значение $2.991\times{}10^{10}$\,см/с. Эти два значения совпадают с точностью до экспериментальной погрешности. Именно этот экспериментальный факт заставил Максвелла предположить, что свет распространяется согласно тем же законам, что и электромагнитные волны. 

Максвелл приписывал особое физическое значение векторному и скалярному потенциалам, $\mathbf{A}$ и $\psi$, которые играют главную роль в его уравнениях. Он так же предполагал наличие механической эфира для объяснения существования тока смещения в свободном пространстве; это предположения вызывало сильное сопротивление теории Максвелла со стороны многих физиков того времени. Уравнения Максвелла, в современной их форме, не содержат никаких потенциалов, как и теория электромагнетизма не нуждается в предположении существования вымышленной среды для поддержания тока смещения в свободном пространстве. Оригинальная интерпретация тока смещения, данная Максвеллом, больше не используется, однако, сам термин остался в память о заслугах Максвелла. 

Дальше, Максвелл предположил, что плоская волна распространяется сквозь пространство со скоростью $\mathbf{V}$ в направлении, определяемым единичным вектором $\mathbf{w}$, чьи направляющие косинусы в направлениях $x,\,y,\,z$ обозначены как $l,\,m,\,n$, соответственно. В этом случае, электромагнитная волна будет функцией параметра
\begin{align*}
w=lx+my+nz-Vt
\end{align*}
Используя выражение для напряженности магнитного поля~\eqref{eq:B}, Максвелл показывает, что:
\begin{align*}
\mu\mathbf{H}\perp\mathbf{w},
\end{align*}
что выражает тот факт, что вектор напряженности магнитного поля перпендикулярен направлению распространения волны, т.е. лежит в плоскости волны. Рассматривая непроводящую и стационарную среду ($\mathbf{J}\equiv0$), из уравнений~\eqref{eq:B}, \eqref{eq:C} и ~\eqref{eq:D} Максвелл получает, что:
\begin{align}
\mathrm{k}\left[\nabla\left(\nabla\cdot\mathbf{A}\right)-\nabla^2\mathbf{A}\right]+4\pi\mu\left[\frac{\partial^2\mathbf{A}}{\partial{}t^2}+\nabla\left(\frac{\partial\psi}{\partial{}t}\right)\right]=0.\label{eq:5.4}
\end{align}
Затем, Максвелл выражает $\mathbf{A}$ и $\psi$ из уравнения~\eqref{eq:5.4} и получает уравнение для вектора $\mathbf{H}$ как функции $\mathbf{w}$:
\begin{align*}
\mathrm{k}\mu\frac{\operatorname{d}^2\mathbf{H}}{\operatorname{d}\mathbf{w}^2}=4\pi\mu^2V^2\frac{\operatorname{d}^2\mathbf{H}}{\operatorname{d}\mathbf{w}^2},
\end{align*}
что математически выражает тот факт, что волна распространяется со скоростью\begin{align}
V=\pm\sqrt{\frac{\mathrm{k}}{4\pi\mu}}\label{eq:5.7}
\end{align} 
в направлениях $\pm\mathbf{w}$.
Максвелл дает здесь свой комментарий:
\begin{quotation}
\small
Эта волна состоит в возмущении магнитного поля, направление поляризации которого лежит в плоскости волны. Возмущения магнитного поля, чья поляризация лежит не в плоскости волны, не могут распространяться в виде плоской волны.

Поскольку, магнитные колебания распространяются в электромагнитном поле со скоростью света, таким образом, что возмущение в любой точке перпендикулярно направлению распространения, эти волны могут иметь все свойства, присущие поляризованному свету.
\end{quotation}

Несмотря на то, что Максвелл изначально рассматривал только возмущения магнитного поля, позже он показывает, что $\mathbf{E}\perp\mathbf{w}$ и $\mathbf{E}\perp\mathbf{H}$. Таким образом, возмущения электрического и магнитного полей  взаимно ортогональны и  лежат в плоскости распространяющейся плоской волны:
\begin{quote}\small
Математическая форма записи возмущения, таким образом, согласуется с тем, что возмущения, составляющие свет, перпендикулярны направлению своего распространения.
\end{quote}
Подставляя выражения для $\mathrm{k}$, данное в~\eqref{eq:5.1}, в уравнение~\eqref{eq:5.7}, получаем, что:
\begin{align*}
V=\frac{C}{\sqrt{K\mu}}
\end{align*}
Максвелл рассматривал распространение волн в свободном пространстве, для которого $K=1$ и $\mu=1$ (в гауссовой системе единиц), поэтому:
\begin{align*}
V=C=(ед.\,эл.\,заряд\,СГСМ)/(ед.\,эл.\,заряд\,СГСЭ)
\end{align*}
Прямые электромагнитные измерения показали, что:
\begin{align*}
C=314\,740\,000\, м/с=c=скорость\,света\,в\,свободном\,пространстве
\end{align*}
После этого Максвелл делает свое наиболее значительное предположение:
\begin{quote}
Это совпадение, по-видимому, показывает, что свет и магнетизм --- суть одно и то же, и что свет --- это электромагнитное возмущение, распространяющееся согласно законам электромагнетизма.
\end{quote} 

Важно отметить, что Максвелл не дал никаких комментариев по поводу возбуждения электромагнитных волн и света прямыми электромагнитными методами; более того, нет ни одного указания, что он рассматривал такую возможность. Максвелл не дожил до того времени, когда его теория подтвердилась экспериментами и была принята учеными. 
\section{Генрих Герц и Максвеллианцы}
Теория Максвелла, в том виде, в котором он оставил ее после своей смерти в 1879~г., рассматривалась научным сообществом того времени как одна из многих попыток объяснить распространение света в пространстве и, тем более, никоим образом не претендовала на лидерство в своей области. Но уже к 1890 г. она смела своих конкурентов и по праву заняла место одной из самых удачных и фундаментальных теорий во всей физике. Это стало возможным, во многом, благодаря работам Генриха Герца и группы ``Максвеллианцев'', среди которых особо выделялись Фитцжеральд, Лодж и Хевисайд. Эти ученые, на протяжении 1880-х~гг., переформулировали теорию в более четком и компактном виде, подвергли ее экспериментальной проверке  и обобщили на случаи, которые сам Максвелл не мог предвидеть. В ходе этой работы, теория стала ближе к существующей на тот момент технологии, в частности, в работах Хевисайда, а так же - Лоджа и Герца, давших начало радиосвязи.

Первые упоминания о максвеллианцах стали появляться в 1878~г., когда Лодж и Фитцежеральд впервые встретились и обнаружили, что оба они полны энтузиазма относительно \emph{Трактата} Максвелла, не смотря на то, что, по их же признанию, у них нет полного понимания всей работы. По словам Фитцжеральда, он изначально предполагал, что теория Максвелла исключает возможность прямой генерации электромагнитных волн, и в 1879~г. он отговаривает Лоджа от попыток их экспериментального возбуждения. Фитцжеральд вскоре осознал свою ошибку и уже в 1883~г. публикует работу, в которой описывает методику генерации электромагнитных волн метрового диапазона при помощи разряда конденсатора через малое сопротивление. Но ни сам Фитцежеральд, ни Лодж не смогли придумать способа детектирования столь быстрых электрических колебаний, и, к середине 1880-х они бросили всякие попытки добиться этого.

Позже, в 1883~г., английский ученый Джон Пойнтинг обнаружил, что, согласно теории Максвелла, энергия электромагнитной волны должна распространяться вдоль направления, перпендикулярном как электрическому, так и магнитному полям. Из теоремы Пойнтинга следовало, что энергия электрического тока переносится не по самому кабелю, как было принято думать, а через пространство вокруг него. Фитцжеральд и Лодж вскоре осознали, что теорема Пойнтинга --- ключ к пониманию теории Максвелла, несмотря на то, что сам Максвелл не догадывался об этом явлении. В 1885~г. Фитцжеральд описал  механическую модель, состоящую из медных колес и резиновых ремней передачи, для иллюстрации того, как энергия переносится в эфире, одновременно с  Лоджем, использовавшим похожую модель для описания передачи энергии через пространство в своей работе \emph{Современные представления об электричестве} (1889~г.).

Хевисайд, эксцентричный бывший кабельный инженер, ушедший в ``отставку'' в возрасте 24 лет дабы посвятить себя теории электричества, независимо от Пойнтинга пришел к теореме о плотности потока энергии, которую он также рассматривал как центральную в теории Максвелла. Будучи убежденным, что использованный в оригинальных уравнениях Максвелла вектор потенциал усложняет истинную природу распространения энергии в электромагнитном поле, Хевисайд переписывает уравнения, данные в \emph{Трактате} в компактной форме системы из четырех векторных уравнений, известных сейчас как ``уравнения Максвелла'':
\begin{align*}
&\operatorname{div}\varepsilon{}\mathbf{E} = \rho& 
&\operatorname{rot}\mathbf{H}=\mathrm{k}\mathbf{E} + \varepsilon{}d\mathbf{E}/dt\\
&\operatorname{div}\mu{}\mathbf{H}=0& 
-&\operatorname{rot}\mathbf{E}=\mu{}d\mathbf{H}/dt
\end{align*}
эти уравнения ведут к известному выражению для плотности потока энергии ($\mathbf{S}=\mathbf{E}\times\mathbf{H}$) и, в простой и ясной форме, выражают многие другие аспекты теории Максвелла.

В середине 1880-х, Хевисайд применяет расширенную материальными связями систему уравнений для исследования вопроса распространения электромагнитных волн вдоль электрической линии --- основную задачу телеграфии. Согласно выводам Хевисайда, сигнал распространяется не внутри провода, но ``скользит'' вдоль него в окружающем пространстве. В 1886~г. он аналитически показывает, что затухание сигнала может быть значительно уменьшено, даже --- устранено полностью, если нагрузить линию специально подобранной катушкой индуктивности. Метод индуктивной нагрузки, в дальнейшем, показал свою высокую эффективность в телефонной и кабельной индустрии, но сразу же после публикации своего изобретения, Хевисайд встретил жесткий отпор со стороны бывшего в то время главы Британской Почтовой службы, Вильяма Приса. Прис считал всякую индуктивность в телеграфной линии вредной для распространяющегося сигнала (что, конечно же, верно в случае неверно подобранного значения нагрузочной индуктивности) и предпринял успешные попытки заблокировать публикацию противоречащей его взглядам работы Хевисайда. Прочие важные работы Хевисайда по вопросам теории Максвелла и распространения электромагнитных волн, не успевшие привлечь к себе внимания, стали так же преследоваться Присом.

Для Хевисайда было большой удачей, что, начиная с 1888~г., экспериментальные исследования Лоджа в Англии и Герца в Германии, стали привлекать все больше внимания со стороны научного сообщества. Пытаясь получить в своей лаборатории молниеподобные искры посредством быстрой разрядки больших конденсаторов через проволочные перемычки, Лодж обнаружил явление возбуждения электромагнитных волн, которые ``скользят'' вдоль перемычек в окружающем их пространстве, как то предсказывала теория телеграфа Хевисайда. Вскоре после этого открытия, Лодж и Хевисайд вступили в переписку и начали совместную работу над теорией Максвелла. Затем, в середине 1888~г., из Германии пришли новости о еще более впечатляющих экспериментах Герца с электромагнитными волнами в воздухе.  В Берлине Герц учился у Гемгольца, от которого узнал о модифицированной версии уравнений Максвелла, и, вдохновившись ими, решил испытать их против дальнодействующих теорий Вебера и Ньюмана. Умея получать короткие электрические вспышки при помощи разряда конденсаторов через воздушные зазоры, Герц сумел зарегистрировать интерференцию возбуждаемых таким образом электромагнитных волн метрового диапазона в  лекционном зале Карлсруэ и измерить их основные параметры.

Эксперименты Герца были тепло встречены британскими максвеллианцами, которые рассматривали эти результаты как долгожданное подтверждение собственных теоретических предсказаний. В сентябре 1888~г., на встрече Британской Ассоциации, Фитцжеральд заявил, что ``великолепные'' эксперименты Герца окончательно доказали, что электромагнитные силы не являются дальнодействующими. Лодж, который вместе с Хевисайдом, был близок к открытию, присоединился к поздравлениям, заметив, с иронией, что столь неопровержимое доказательство близкодействующей теории поля Максвелла пришло из Германии --- великой родины дальнодействующих теорий. Континентальные физики вскоре подхватили теорию Максвелла, или, по-крайней мере, модифицированную версию ее уравнений, близких к уравнениям Хевисайда, опубликованных Герцом в 1890~г. Окрепшая теория Максвелла вошла в 1890~г. готовой поглотить в себя не только оптику, но и многие другие направления в физике. По словам Лоджа, учение об электричестве стало ``имперской наукой''.

Учение об электричестве было имперской наукой в прямом смысле этого слова: подводные кабели, опоясывавшие практически весь мир, были ``нервной системой'' Британской империи. Замечателен тот факт, в свете тесной связи Британской кабельной индустрии и максвеллианской теории, последняя послужила началом новой технологии --- радиосвязи --- которая, в скором времени, разрушила монополию Британии на глобальные телекоммуникации. Герц был сфокусирован на исследовании свойств электромагнитных волн и не рассматривал их как возможное средство сообщения.  Но в 1894~г., Лодж, без всякого на то намерения, случайно изобрел метод радиосообщения при помощи азбуки Морзе. Вслед за ним и независимо друг от друга, Маркони в Италии и Попову в России, удалось собрать первые образцы радиопередатчиков, способных применяться на практике. Беcпроводной телеграф стал широко использоваться после 1900~г., дополняя и, даже, конкурируя с кабельной сетью. Изобретение передатчиков непрерывной волны и вакуумных усилительных трубок преобразило индустрию радиосвязи, и первые эксперименты по широковещанию в 1920-х годах перенесли взаимоотношения науки об электричестве и технологии на более высокий уровень.
\section{Технический прогресс и теория Максвелла}
В вводном слове к своему \emph{Трактату об электричестве и магнетизме}, Максвелл отмечает, что 
\begin{quote}
\small
многочисленные приложения, которые нашла теория электромагнетизма в телеграфии, в конечном итоге, повлияли на развитие чистой науки в том смысле, что составили коммерческую ценность точным электрическим измерениям и предоставили заинтересованным ученым техническое оснащение в масштабах и количестве, недостижимом для рядовой лаборатории. Подобная заинтересованность в знаниях об электричестве и, одновременно, экспериментальная возможность получить их, уже дали свои первые плоды, как в стимулировании усилий людей науки, так и повсеместном распространении точного знания среди практиков, что, в целом, способствовало  всеобщему научному прогрессу во всей инженерной профессии.
\end{quote}
Максвелл, очевидно, подразумевает здесь свою работу в Комитете по Стандартам и, конечно же, участие Томсона в работе над проектом прокладки второго Атлантического кабеля, законченную в 1866~г.

Спустя несколько лет после провала в 1858~г., Филд и партнеры предпринимают вторую попытку проложить кабель по дну Атлантического океана. Следуя рекомендациям, данным Томсоном, уже более последовательно в этот раз, партнеры заказывают более тонкий кабель и тщательно его проверяют. В июле 1865~г. корабль \emph{Great Eastern} --- единственный на тот момент во всем флоте, способный нести на борту такое количество кабеля (4260 км), начинает прокладку от Ирландии, но, к сожалению, кабель обрывается, когда было проложено около 1968 км, а его конец теряется. Проект кажется обреченным на провал, но, при поддержке Джона Пендера, богатого торговца хлопком из Манчестера, Филд и партнеры заказывают еще один кабель и следующим летом предпринимают уже третью попытку. В этот раз все проходит гладко и 27 июля 1866~г. партнеры закрепляют кабель вблизи Ньюфаундленда. Вскоре после этого, \emph{Great Eastern} находит ранее утерянный конец кабеля, оставшийся от неудачной попытки 1865~г., сращивает его и заканчивает прерванные работы по прокладке, и, таким образом, с сентября 1866~г. Атлантика стянута сразу двумя рабочими кабелями.

Успех 1866~г. вызвал всеобщий бум в прокладке кабеля. К 1875~г., британские компании (владельцем большинства из них был тот самый Пендер), проложили кабели к Индии, Австралии, Гонконгу; к 1890~г. кабели опоясывают побережья Южной Америки и Африки, и обеспечивают связью уже весь мир. Телеграфия становится огромной и прибыльной индустрией, сильно заинтересованной в квалифицированных электриках на всем протяжении 1850--1880~гг. Британские фирмы владели на тот момент примерно двумя третями от общей длины проложенного кабеля, что способствовало повышенной заинтересованности британских ученых в исследовании вопросов распространения электромагнитных сигналов в линиях.

В конце 1870-х, стали появляться первые электрические станции, ознаменовав, тем самым, рождение электротехнической промышленности и вызвав новую волну спроса на исследования в области электричества. Электроэнергия, получаемая от батарей Вольта, или от генераторов на постоянных магнитах, была слишком слабой и дорогостоящей для широкого использования. В 1866--1867~гг. несколько изобретателей предлагают использовать в генераторах электромагниты, что существенно увеличивает их мощность и делает электроэнергию, на первое время, относительно дешевой и доступной. Первым основным применением электроэнергии, получаемой от динамо-машин, были осветительные дуги, но скоро последние стали слишком яркими для использования их в помещениях. Томас Эдисон (1816--1890), изобретает лампу накаливания в 1879~г. и приступает к разработке системы производства и распределения электрической мощности; в 1882~г. он устанавливает первые электрические станции на Pearl Street в Нью-Йорке.

Эдисон изобрел систему распределения постоянного тока, которая подходила для передачи электроэнергии только на короткие расстояния вследствие сильных потерь в линии. Системы распределения переменного тока были лишены этого недостатка, используя повышающие трансформаторы перед передачей электроэнергии на удаленные подстанции, и понижающие --- для окончательного распределения мощности между местными потребителями. В нашумевшей ``битве систем'', длившейся вплоть до 1890~г., победителями вышли системы переменного тока.

Ранние системы распределения постоянного тока были примитивными, электрический ток в их проводах мог, в большинстве случаев, рассматриваться как поток воды в трубопроводе. Системы распределения переменного тока были уже тогда более сложными (в частности, после изобретения метода полифазной передачи), в которых наблюдались многие эффекты, связанные с переменными электромагнитными полями. Стремительный рост электротехнической промышленности в 1880-х и 1890-х годах способствовал беспрецедентному росту потребности в квалифицированных инженерах-электриках, компетентных для работы с переменным током. Этот спрос нашел отклик в физических факультетах университетов по всему миру, которые стали открывать расширять свои кафедры и лаборатории, чтобы справиться с нахлынувшим потоком студентов. Можно с уверенностью сказать, что развитие физики как дисциплины во второй половине девятнадцатого столетия во многом связан со спросом и возможностями, предоставляемыми электротехнической промышленностью. 

\section{Начало унификации физических теорий}
Карл Гаусс, своими работами о земном магнетизме, произвел настоящую революцию в электродинамике и физике.  На протяжении 1832~г. Гаусс работал над разработкой и экспериментальной проверкой метода измерения величины и ориентации магнитного поля Земли, который бы не зависел от характеристик измерительного компаса. Методики, используемые до того времени, были ненадежными в силу сильной зависимости результатов измерения от магнитного момента измерительной стрелки. Из за своих основных преимуществ (независимость от измерительного инструмента и независимость от положения), Гаусс назвал величины, использованные им при описании теории, ``абсолютными''. Вильгельм Вебер  был другом и соавтором Гаусса в университете Готингема, поэтому истоки исследовательских работ Вебера по вопросам электричества следует искать в исследованиях магнетизма Гауссом. Вебер и Гаусс стали исследовать вопрос применения техник измерения магнитного поля в новой области, связанной с открытием Фарадеем явления электромагнитной индукции. В 1843~г., Вебер заинтересовался электродинамикой Ампера, и, уже в 1846~г., опубликовал свои важные научные результаты в этой области в работе ``Электромагнитные измерения основного закона электрического действия''. Закон, описанный Вебером в этой работе, был фундаментальным в том смысле, что электрическое действие, приложено к самой электрической ``массе'', а не к ее переносчику --- проводникам. 

Определение электродинамических и электромагнитных единиц плотности тока, полученное независимо от Гаусса и основанное на взаимодействии ``ток--магнитное поле'', было включено в собрание работ Вебера и представляет собой основу для абсолютных измерений токов в терминах механических величин.

Для того, чтобы выявить физический смысл фигурирующей в уравнении фундаментального закона константы $c$, имеющей размерность скорости, Вебер описывает смысл всех символов, входящих в его теорию. Эти символы представляют собой рациональные числа, т.е. значения неких физических величин. Вебер точно (для того времени) измерил значение $c$ и нашел отношение между силой, оказываемой данным количеством электричества трения, накопленного  конденсаторе, и силой, оказываемой им же, движущемся по проводнику. Этим он решил давно известную проблему связи электростатики и электродинамики, проблему, которую, в свое время, обошел Ампер, и заинтриговавшую Фарадея в его исследованиях. Вебер отметил, что в его фундаментальном законе $c^2$ представляет собой отношение между электростатической (``заряд--заряд'') и электродинамической (``ток--ток'') силами, измеренное в механических единицах.

Несмотря на то, что немецкие ученые не отождествляли $c$ со скоростью света, как это сделал Максвелл в 1862~г., Вебер отвел $c$ роль второй универсальной константы в теории Ньютона. Помимо вышеизложенной роли $c$ в теории, Вебер также приписывал этой константе метрологическое значение, как проявление связи между единицами времени и пространства, что позволило оставить только две фундаментальные физические величины (пространство и массу).

Стоит заметить, что достижения Гаусса и Вебера стали возможными во многом благодаря использованным ими метрологическому подходу. Этот подход имел большое влияние на методы физики. Фактически, благодаря систематизации величин в систему абсолютных единиц, законы электродинамики могли быть записаны в виде аналитических уравнений, включающих физически важные константы пропорциональности. Исторически, это представление было разработано и стало стандартным в физике: уравнения заменили соотношения пропорциональности и безразмерные величины, форму, в которой физические законы были записаны в начале девятнадцатого века в работах Кулона, Френеля и других. Это, много более богатое представление физических законов, имело то важное методологическое последствие, что теории могли быть использованы для предсказания результатов экспериментов, в которых измеряются абсолютные значения физических величин. Генрих Герц, спустя несколько десятилетий, рассматривал предсказательную силу теорий как основное свойство теоретической физики.

Считается, что Максвелл, совместно с Кельвином и Гамильтоном, ввел новую символику в математическое описание физических законов. Символы в уравнениях не являлись теперь просто числами, но имели размерность. Размерности имели важную роль в основном труде Максвелла ``Трактат об электричестве и магнетизме'', т.е. для его электромагнитной теории света, которая была поддержана, в том числе, доказательством, напрямую связанным с использованным метрологическим подходом. 

В своей статье ``О математической классификации физических величин'', Максвелл описал новый способ связать математику и физику: его классификация основывалась на математических (формальных) аналогиях между физически разными величинами. Этот подход отличается от традиционного способа классификации величин, основанном на разделении величин относительно физических явлений, к которым они принадлежат. Новая классификация Максвелла имела то преимущество, что позволяла проводить аналогии между величинами, принадлежащим различным явлениям, таким как гравитация и теплопроводность, так, что ``одна теория может быть использована для решения задач в другой области''. Величины были классифицированы в соответствии с их векторным или тензорным представлением, т.е., с их математическими свойствами, что позволило, в той мере, как это позволяет математика, проводить обобщения физических законов. 

Классификация Максвелла подразумевает новый смысл, придаваемый физическим символам: каждый символ состоит из двух частей --- численное значение, умноженное на единицу физической величины. В случае энергии, Максвелл отметил, что эта величина может быть выражена двумя различными способами: как квадрат скорости, помноженный на массу, или как  произведение момента движения и скорости, где оба множителя --- векторные величины. Второе определении оказалось наиболее успешным в новой науке, потому что оно позволяло физическую интерпретацию: ``первый множитель понимается как стремление к определенному изменению, а второй --- само изменение''. Для того, чтобы привести свою математическую классификацию к внутреннему порядку, Максвелл ввел новые символы и обозначения (такие, как векторные операторы ротора, дивергенции, и т.д.). Хевисайд и Гибс рассматривали математику, использованную в ``Трактате'', как первое применение векторного анализа в физике.

\section{Заключение}
Само рождение теории стало возможным благодаря новаторскому способу математической записи физических законов, изобретенной Гауссом и Вебером в виде аналитических уравнений, описывавших непосредственно абсолютные значения физических величин. Фигурирующая в уравнениях Максвелла константа $c$, имевшая размерность скорости, и равная по величине скорости света в свободном пространстве, натолкнула его на мысль о связи между явлениями электромагнетизма и распространением света. Развитие математических приемов описания физических величин в трудах Максвелла, дало толчок развитию унификации физических теорий.

Исторически, появление теории совпало со временем  бурного развития тех индустрий (проводная телеграфия, электрическая промышленность и т.д.), остро нуждавшихся в квалифицированных электриках, инженерах, способных работать с сетями переменного тока, и, соответственно, научных разработках в этих областях. Это способствовало, как спонсированию коммерческими организациями видных ученых, занимающихся проблемами электричества, так и расширению  физических кафедр и факультетов в университетах мира, что, в целом, дало мощный толчок развитию физического знания.

Связав феномен распространения электромагнитных сил в пространстве и явление волнового распространения света, Максвелл, тем самым, подсказал метод экспериментальной проверки своей теории. Однако, математическая запись теории, которую он оставил ее будущим исследователям, нуждалась в значительной переработке, затянувшейся более чем на двадцать лет.  Как результат этой работы, Генрихом Герцом в 1890~г. были опубликованы уравнения, всемирно известные сегодня как уравнения Максвелла. Именно эта запись уравнений позволила раскрыть истинный физический смысл теории и провести экспериментальную проверку предсказываемых эффектов. 

История развития электромагнитной теории света Максвелла, является примером того, как теоретический интерес исследователей встретился с практической заинтересованностью предпринимателей, что привело  к расширению научного знания в этой области и быстрому развитию методов его практического применения. Этот плодотворный симбиоз, в конечном итоге, изменил как на повседневную жизнь людей, предоставив быстрый способ коммуникации, без которого немыслимо современное общество, так и на развитие научного представления о мире.  
\clearpage
\addcontentsline{toc}{section}{Список литературы} 
\begin{thebibliography}{9}
\bibitem{TheCambridge} Mary~Jo~Nye (ed.), \emph{The Cambridge History of Science: The Modern Physical and Mathematical Sciences}, Cambridge University Press, 1st edition, 2002

\bibitem{Wireless} T.K.~Sarkar and others, \emph{History of Wireless}, Wiley-IEEE Press, 2006

\bibitem{TheMaxwell} P. M. Harman and Peter M. Harman, \emph{The Natural Philosophy of James Clerk Maxwell},  Cambridge University Press, 1998

\bibitem{EarlyHistory}G. R. M. Garratt, \emph{The Early History of Radio: From Faraday to Marconi},  Institution of Engineering and Technology, 1994

\bibitem{Salvo}Salvo D'Agostino, \emph{A History of the Ideas of Theoretical Physics}, Kluwer Academic Publishers, 2000 

\end{thebibliography}
\end{document}
