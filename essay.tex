\documentclass[12pt, oneside, a4paper]{article}
\usepackage{ifpdf}
\usepackage[colorlinks,bookmarksopen]{hyperref}
\usepackage[utf8]{inputenc}
\usepackage[english,russian]{babel}
\begin{document}
\section*{Введение}
В 1864 г. Максвелл, на основе его известных уравнений, предположил, что свет представляет собой поперечные электромагнитные волны. Несмотря на это, сам Максвелл не рассматривал возможность получить свет электромагнитными методами. Впрочем, ничего конкретного, о самих электромагнитных волнах, их генерации и детектировании, в его работах нет. Только спустя почти четверть века с момента опубликования электромагнитной теории света Максвелла, Генрих Герц (Heinrich Hertz) в своих выдающихся  экспериментах зарегестрировал и детально изучил основные свойтсва электромагнитных волн, подтвердив, тем самым, теорию Максвелла. Идеи и уравнений  Максвелла были детально разработаны, модифицированы и облачены в удобную для понимания их физической сути форму усилиями Фитцжеральда (G.F.~FitzGerald), Оливера Лоджа (Oliver Lodge), Оливера Хевисайда (Oliver Heaviside) и Генриха Герца. Первые три ученых образовывали группу, неформально названную Максвеллианцами. 

Данная работа посвящена описанию становления электромагнитной теории света Максвелла как одной из самых ярких обобщающих физических теорий.

\section*{Максвелл}
Джеймс Клерк Максвелл (James Clerk Maxwell, 1831-1879) был одним из выдающихся физиков девятнадцатого столетия. Спектр проблем, охватываемый его работами, очень широк и включает в себя как кинетическую теори газов, так и теорию цветового восприятия, но наибольшую известность принесла Максвеллу его электромагнитная теория света. Несмотря на то, собрания сочинения Максвелла и исследования его работ несколько раз переиздавались, большинство бумаг, относящиеся непосредственно к личной жизни Максвелла были утеряны после его смерти. Наиболее полная изданная на данный момент биография Максвелла относится к 1882 году.

Максвелл родился и вырос в Эдинбурге, в семье, происходившей из юго-западной Шотландии. После окончания Эдинбургского университета, Максвелл поступил в Кембридж, который закончил в январе 1854 с отличием в математике. В поисках темы для своей исследовательской работы, в следующем месяце он написал он послал письмо Томсону (Thomson), в котором говорил о своем интересе к электричеству и просил рекомендации относительно работ Фарадея. Неизвестно, почему Максвелл выбрал электричсетсво, тему, которая была исключена из учебного курса Кемриджа, и почему он примкнул к непризнанному подходу Фарадея, но стоит заметить, что  его письмо Томсону появилось на волне интереса, поднятого лекцией Фарадея о задержках в кабелях (cable retardation), данной в Королевском Институте (Royal Institution). 

Первым результатом исследований Максвелла была работа "О фарадевских силовых линиях", законченная в начале 1856. Проводя аналогию между силовыми линиями электрического и магнитного полей и потоками в жидкостях, Максвелл облек расплывчатые формулировки Фарадея в строгую математическую форму. В 1861 он выступил с амбициозной работой "О физических силовых линиях", основанной на разработанной механической модели эфира, составленного из крошечных вихрей и шестеренок. Эта модель привела Максвелла не только к вравнению, связывающему основыне электромагнитные величины и пониманию, что переменные электрические силы генерируют токи смещения, но также к удивительному выводу, что \emph{свет состоит в поперечных колебаниях той же среды, которая является причиной электрических и магнитных явлений}. 

Электромагнитная теория света по праву считается одной из величайших обобщающих теорий во всей физике, поэтому путь, котоым Максвелл пришел к ней, детально изучен. Единственный вопрос, до сих пор вызывавший споры среди историков и философов науки - это вопрос о том, насколько реальной Максвелл считал свою модель, изложенную в "Физических линиях". Общим мнением счиатетя, что Максвелл действительно расмматривал вихри как реально сущесвтующие в магнитном поле, однако промежуточные шестерни были введены им не более чем для удобства и нагляности теоритеского рассмотрения. 

Лучшим доказетльсвом электромагнитной теории Максвелла было совпадение между скоростью света, измеренным Физо (Hippolyte Fizeau) и другими, и отношением между элеткрическими и магнитными системами единиц, измеренным Вебером (Weber) и Рудольфом Колраушем (Rudolph Kohlraush). Более точное измерение отношения единиц должно было стать серьезным испытанием для его теории, и, надежде на это, Максвелл вступает в Британскую Ассоциацию Комитетов по Электрическом Стандартам (British Committee on Electrical Standarts) в 1862. Затем, будучи профессором в Королевском Лондоском Колледже (Kings College London), Максвелл в течении последующих двух лет тесно работает с инженером-кабельщиком Флемингом Дженклином (Fleeming Jenklin) над вычислением этой константы, используя установку, разработанную Томсоном. Максвелл вычисляет искомое соотношение, и, несмотря на некоторые несостыковки, заключает, что оно находится соответсвии со скоростью света, достаточным для того, чтобы убедиться в правоте собственной теории, несмотря на то, что Томсон не принял это доказательство как решающее. 

В декабре 1864 Максвелл представил Королевскому обществу (Royal Society) свою "Динамическую теорию электромагнитного поля", в которой он выводит электромагнитные уравнения не используя какую-либо механическую модель, как "Физическиз линиях", но из общей динамики согласованной системы. Основаясь на фарадеевсих представлениях о зарядах и токах как о побочных свидетельствах состояния окружающих систему полей, Максвелл выражает это состояние через изменение того, что он называет "электромагнитным моментом" (позже переименнованым в вектор-потенциал), и описывает, как энергия распределена в полях. Максвелл по-прежнему верил в существование эфира, но, до тех пор, пока до конца не выяснены детали его внутреннего устройства, Максвелл считал, что лучше формулировать теорию с минимумом предполежений. 

Максвелл покинул Королевский Колледж в 1856 и провел следующие несколько лет в своем поместье в Шотландии за написанием своего "Трактата об электричестве и магнетизме" (1873). Несмторя на что, оно было полно ценными идеями, в целом, труд получился бессвяным и сложным для понимания. К твоему времени, как труд был опубликован, Максвелл вступил в кавендишское профессорство в области экспериментальной физики в Кембридже, основанном в 1871, и занимался организацией новой лаборатории, основной деятельностью которой он рассматривал проведение точных электрических измерений. Максвелл умер от рака 5 ноября 1879, в возрасте 48 лет, во время работы над переизданием своего Трактата. К сожалению, к этому времени, его теория электромагнитного поля не была ни полностью понятой, ни широко распространненой, более того, в некотром смысле она была не законченной.   
\end{document}
